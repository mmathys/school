% ------------------------------------------------------------------------------------------------ %
% FUNKTIONEN VON ZUFALLSVARIABLEN
% ------------------------------------------------------------------------------------------------ %


\subsection{Funktionen von Zufallsvariablen}

Ausgehend von den Zufallsvariablen $X_1,\ldots,X_n$ kann man mit einer Funktion $g:\R^n\rightarrow\R$ eine neue Zufallsvariable $Y = g(X_1,\ldots,X_n)$ bilden.

\begin{example}[Summe, diskret]
F�r die Gewichtsfunktion $p_Z$ der Summe $Z = X + Y$ zweier diskreten Zufallsvariablen $X$ und $Y$ mit gemeinsamer Gewichtsfunktion $p$ erh�lt man
$$
p_Z(z) =
\sum_{x_i \in \mathcal{W}(X)} \P[X=x_i,Y=z-x_i] = 
\sum_{x_i \in \mathcal{W}(X)} p(x_i,z-x_i)
$$
\end{example}

\begin{example}[Summe, stetig]
Sind $X$ und $Y$ stetige Zufallsvariablen mit gemeinsamer Dichte $f$, so ist die Verteilungsfunktion $F_Z$ der Summe $Z = X+Y$ gegeben durch
$$
F_Z =
\displaystyle\int_{-\infty}^\infty \int_{-\infty}^{z-x} f(x,y) \d y \d x \overset{v=x+y}{=}
\int_{-\infty}^z \int_{-\infty}^\infty f(x,v-x) \d x \d v
$$
und somit auch die Dichte
$$
f_Z = \frac{\d}{\d z} F_Z(z) = \int_{-\infty}^\infty f(x,z-x) \d x.
$$
\end{example}


% ------------------------------------------------------------------------------------------------ %
% I.I.D.
% ------------------------------------------------------------------------------------------------ %


\subsubsection{i.i.d Annahme}

Die Abk�rzung \emph{i.i.d.} kommt vom Englischen \emph{independent and identically distributed}. Die $n$-fache Wiederholung eines Zufallsexperiments ist selbst wieder ein Zufallsexperiment. F�r die Zufallsvariablen $X_i$ der $i$-ten Wiederholung wird oft aus Gr�nden der Einfachheit Folgendes angenommen:
\begin{compactenum}[i:]
\item $X_1,\ldots,X_n$ sind paarweise unabh�ngig.
\item Alle $X_i$ haben dieselbe Verteilung.
\end{compactenum}


% ------------------------------------------------------------------------------------------------ %
% SPEZIELLE FUNKTIONEN VON ZUFALLSVARIABLEN
% ------------------------------------------------------------------------------------------------ %


\subsubsection{Spezielle Funktionen von Zufallsvariablen}

Wichtige Spezialf�lle sind die Summe $S_n = \sum_{i=1}^n X_i$ und das arithmetische Mittel $\overline{X}_n = \frac{S_n}{n}$.
\begin{compactenum}
\item Wenn $X_i \sim Be(p)$, dann ist $S_n \sim Bin(n,p)$.
\item Wenn $X_i \sim \mathcal{P}(\lambda)$, dann ist $S_n \sim \mathcal{P}(n\lambda)$.
\item Wenn $X_i \sim \mathcal{N}(\mu,\sigma^2)$, dann ist $S_n \sim \mathcal{N}(n\mu,n\sigma^2)$.
\end{compactenum}

F�r den Erwartungswert und die Varianz gilt allgemein
\begin{center}
\begin{tabular}{ll}
$\E[S_n] = n\E[X_i]$ &
$\var[S_n] = n\var[X_i]$
\end{tabular}
\end{center}


% ------------------------------------------------------------------------------------------------ %