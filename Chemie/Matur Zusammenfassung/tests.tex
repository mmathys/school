% ------------------------------------------------------------------------------------------------ %
% TESTS
% ------------------------------------------------------------------------------------------------ %


\section{Tests}

\subsection{Fehler 1. und 2. Art}

\begin{definition}[Fehler 1. Art] Die Hypothese wird zu Unrecht abgelehnt, d.h. obwohl sie richtig ist. Die Wahrscheinlichkeit f�r einen Fehler 1. Art ist
$$
\P_\theta[T \in K]
\quad\text{f�r } \theta \in \Theta_0.
$$
\end{definition}

\begin{definition}[Fehler 2. Art]
Die Hypothese wird akzeptiert, obwohl sie falsch ist. Die Wahrscheinlichkeit f�r einen Fehler 2. Art ist
$$
\P_\theta[T \notin K] =
1 - \P_\theta[T \in K]
\quad\text{f�r } \theta \in \Theta_A.
$$
\end{definition}


% ------------------------------------------------------------------------------------------------ %
% M�GLICHES VORGEHEN
% ------------------------------------------------------------------------------------------------ %


\subsection{M�gliches Vorgehen}

Ausgangspunkt ist eine Stichprobe $X_1,\ldots,X_n$ in einem Modell $\P_\theta$ mit unbekanntem Parameter $\theta \in \Theta$.

\begin{compactenum}[1:]
\item
Aufgrund einer Vermutung, wo sich der richtige Parameter $\theta$ befindet, werden eine \emph{Hypothese} $\Theta_0 \subseteq \Theta$ und eine \emph{Alternative} $\Theta_A \subseteq \Theta$ mit $\Theta_0 \cap \Theta_A = \emptyset$ formuliert:
\begin{center}
\vspace{1ex}
\begin{tabular}{r@{ : }l}
Hypothese $H_0$ & $\theta \in \Theta_0$ \\
Alternative $H_A$ & $\theta \in \Theta_A$
\end{tabular}
\end{center}
\end{compactenum}

\begin{note}
Die Hypotese (bzw. Alternative) heisst \emph{einfach}, falls sie nur aus einem einzelnen Wert besteht, also z.B. $\Theta_0 = \{\theta_0\}$ (bzw. $\Theta_A = \{\theta_A\})$.
\end{note}

\begin{compactenum}
\setcounter{enumi}{1}
\item
Es wird eine \emph{Teststatistik} $T = t (X_1,\ldots,X_n)$ gew�hlt, wobei $t:\R^n\rightarrow\R$ eine geeignete Funktion ist.

\item
Es wird ein \emph{Signifikanzniveau} $\alpha \in (0,1)$ gew�hlt.

\item Ein \emph{Verwerfungsbereich} $K \subseteq \R$ wird konstruiert, so dass
$$
\sup_{\theta \in \Theta_0} \P_\theta[T \in K] \leq \alpha.
$$
Dadurch wird die Wahrscheinlichkeit eines Fehlers 1. Art durch $\alpha$ beschr�nkt.

\item
Die Hypothese wird verworfen, falls der realisierte Wert $t(x_1,\ldots,x_n)$ im Verwerfungsbereich $K$ liegt.
\end{compactenum}

\begin{note}
Alternative zu Schritt 4 und 5:
Der P-Wert $p$ wird berechnet und die Hypothese verworfen, falls $p \leq \alpha$.
\end{note}

\begin{definition}[P-Wert]
Der P-Wert ist die Wahrscheinlichkeit, dass unter der Nullhypothese $H_0$ ein zuf�lliger Versuch mindestens so extrem ausf�llt, wie der beobachtete Wert $t$.
\end{definition}

\begin{definition}[Macht]
Die \emph{Macht} eines Tests ist die Funktion
$$
\beta:\Theta_A\rightarrow[0,1],\quad
\theta \mapsto\beta(\theta) := \mathbb{P}_\theta[T \in K].
$$
Das Maximieren der Macht $\beta(\theta)$ entspricht dem Minimieren der Wahrscheinlichkeit f�r einen Fehler 2. Art $1-\beta(\theta) = \mathbb{P}_\theta[T\notin K]$ f�r $\theta \in \Theta_A$.
\end{definition}


% ------------------------------------------------------------------------------------------------ %
% LIKELIHOOD QUOTIENT
% ------------------------------------------------------------------------------------------------ %


\subsection{Likelihood-Quotienten Test}

Als Teststatistik wird der \emph{Likelihood-Quotient} $\mathcal{R}$ gew�hlt, wobei $\mathcal{L}$ die Likelihood-Funktion ist:
$$
T :=
\mathcal{R}(x_1,\ldots,x_n) :=
\frac
  {\displaystyle\sup_{\theta\in\Theta_0}\mathcal{L}(x_1,\ldots,x_n \mid \theta)}
  {\displaystyle\sup_{\theta\in\Theta_A}\mathcal{L}(x_1,\ldots,x_n \mid \theta)}
$$
Ist dieser Quotient klein, sind die Beobachtungen im Modell $\P_{\Theta_A}$ deutlich wahrscheinlicher als im Modell $\P_{\Theta_0}$. Der Verwerfungsbereich $K := [0,c)$ wird so gew�hlt, dass der Test das gew�nschte Signifikanzniveau einh�lt.

\begin{note}
Sind Hypothese und Alternative beide einfach, so ist der Test optimal (nach Neyman-Pearson-Lemma).
\end{note}


% ------------------------------------------------------------------------------------------------ %
% Z-TEST
% ------------------------------------------------------------------------------------------------ %


\subsection{z-Test}

%Normalverteilung, Test f�r Erwartungswert bei bekannter Varianz:
Seien $X_1,\ldots,X_n \overset{i.i.d.}{\sim} \mathcal{N}(\theta_0,\sigma^2)$ unter $\P_{\theta_0}$ mit \emph{bekannter} Varianz $\sigma^2$. Es soll die Hypothese $H_0: \theta = \theta_0$ getestet werden. M�gliche Alternativen $H_A$ sind $\theta > \theta_0$, $\theta < \theta_0$ (einseitig) oder $\theta \neq \theta_0$ (zweiseitig). Die Teststatistik ist
$$
T := \frac{\overline{X}_n - \theta_0}{\sigma_X / \sqrt{n}} \sim \mathcal{N}(0,1)
$$
unter dem Modell $\P_{\theta_0}$.  Der Verwerfungsbereich ist von der Form $(c_>,\infty)$, bzw. $(-\infty,c_<)$, bzw. $(-\infty,-c_{\neq})\cup(c_{\neq},\infty)$. Zum Beispiel liefert die Bedingung
$$
\alpha =
\P_{\theta_0}[T \in K_>] =
\P_{\theta_0}[T > c_>] =
1- \Phi(c_>),
$$
dass $c_> = \Phi^{-1}(1-\alpha)$, also das $(1-\alpha)$-Quantil der $\mathcal{N}(0,1)$-Verteilung, sein muss.


% ------------------------------------------------------------------------------------------------ %
% T-TEST
% ------------------------------------------------------------------------------------------------ %


\subsection{t-Test}

%Normalverteilung, Test f�r Erwartungswert bei unbekannter Varianz:
Seien $X_1,\ldots,X_n \overset{\text{i.i.d.}}{\sim} \mathcal{N}(\mu_0,\sigma^2)$ unter $\P_\theta$ wobei $\theta = (\mu,\sigma^2)$ und insbesondere die Varianz $\sigma^2$ \emph{unbekannt} ist. Die Hypothese $H_0: \mu = \mu_0$ soll getestet werden.
Die unbekannte Varianz $\sigma^2$ wird durch den Sch�tzer $s^2 = \frac{1}{n-1} \sum_{i=1}^n (X_i-\overline{X})^2$ (empirische Varianz) ersetzt. Danach kann mit der Teststatistik
$$
T := \frac{\overline{X}_n - \mu_0}{s / \sqrt{n}} \sim t_{n-1}
$$
gleich wie beim z-Test vorgegangen werden.


% ------------------------------------------------------------------------------------------------ %
% GEPAARTE ZWEISTICHPROBEN TESTS
% ------------------------------------------------------------------------------------------------ %


\subsection{Gepaarter Zweistichprobentest}

Seien $X_{1\leq i \leq n} \overset{\text{i.i.d.}}{\sim} \mathcal{N}(\mu_X,\sigma^2)$ und $Y_{1\leq i \leq n} \overset{\text{i.i.d.}}{\sim} \mathcal{N}(\mu_Y,\sigma^2)$ unter $\P_\theta$. Falls man eine nat�rliche Paarbildung zwischen $X_i$ und $Y_i$ hat, l�sst der Test zum Vergleich von $\mu_X$ und $\mu_Y$ auf eine Stichprobe zur�ckf�hren:
$$
Z_i := X_i - Y_i
\overset{\text{i.i.d.}}{\sim}
\mathcal{N}(\mu_x-\mu_y,2\sigma^2)
$$

\subsection{Ungepaarter Zweistichprobentest}

Seien $X_{1\leq i \leq n} \overset{\text{i.i.d.}}{\sim} \mathcal{N}(\mu_X,\sigma^2)$ und $Y_{1\leq i \leq m} \overset{\text{i.i.d.}}{\sim} \mathcal{N}(\mu_Y,\sigma^2)$ unter $\P_\theta$.

\begin{compactenum}[\bf a)]
\item
	Ist $\sigma^2$ bekannt, so ist die Teststatistik
	$$
	T := \frac{(\overline{X}_n-\overline{Y}_m)-(\mu_X-\mu_Y)}{\sigma\sqrt{\frac{1}{n}+\frac{1}{m}}} \sim \mathcal{N}(0,1).
	$$
\item
	Ist $\sigma^2$ unbekannt, berechnet man
	$$
	s^2 := \frac{1}{m+n-2}((n-1)s_X^2+(m-1)s_Y^2)
	$$
	und w�hlt f�r die Teststatistik
	$$
	T := \frac{(\overline{X}_n-\overline{Y}_m)-(\mu_X-\mu_Y)}{s\sqrt{\frac{1}{n}+\frac{1}{m}}} \sim t_{n+m-2}
	$$
\end{compactenum}


% ------------------------------------------------------------------------------------------------ %
% VERTRAUENSINTERVALL
% ------------------------------------------------------------------------------------------------ %


\subsection{Konfidenzbereiche}

\begin{definition}[Konfidenzbereich]
Ein \emph{Konfidenzbereich} f�r $\theta$ zu den Stichproben $X_1,\ldots,X_n$ ist eine Menge $C(X_1,\ldots,X_n) \subseteq \Theta$. In den meisten F�llen ist das ein Intervall, dessen Endpunkte von $X_1,\ldots,X_n$ abh�ngen.

$C$ heisst ein Konfidenzbereich zum \emph{Niveau} $1-\alpha$, falls gilt
$$
\P_\theta[\theta \in C(X_1,\ldots,X_n)] \geq 1-\alpha
$$
\end{definition}


% ------------------------------------------------------------------------------------------------ %