% ------------------------------------------------------------------------------------------------ %
% Zufallsvariablen
% ------------------------------------------------------------------------------------------------ %


\section{Zufallsvariablen}

\begin{definition}[Zufallsvariable]
Eine \emph{Zufallsvariable} $X$ auf $\Omega$ ist eine Funktion $X:\Omega \rightarrow \mathcal{W}(X) \subseteq \R$. Jedes Elementarereignis $\omega$ wird auf eine Zahl $X(\omega)$ abgebildet.
\end{definition}

\begin{definition}[Verteilungsfunktion]
Die \emph{Verteilungsfunktion} einer Zufallsvariable $X$ ist die Abbildung $F_X:\R \rightarrow[0,1]$,
$$
F_X(t) := \P[X \leq t] := \P[\{\omega \mid X(\omega) \leq t\}].
$$
\end{definition}

Jede Verteilungsfunktion $F_X$ hat folgende Eigenschaften:
\begin{compactenum}[i:]
\item $a \leq b \Rightarrow F_X(a) \leq F_X(b)$ (monoton wachsend).
\item $\displaystyle\lim_{t\rightarrow u, t > u}F_X(t) = F_X(u)$ (rechtsstetig).
\item $\displaystyle\lim_{t\rightarrow-\infty} F_X(t) = 0$ und $\displaystyle\lim_{t\rightarrow\infty} F_X(t) = 1$.
\end{compactenum}


% ------------------------------------------------------------------------------------------------ %
% Diskrete Zufallsvariablen
% ------------------------------------------------------------------------------------------------ %


\subsection{Diskrete Zufallsvariablen}

Eine Zufallsvariable heisst \emph{diskret}, falls ihr Wertebereich $\mathcal{W}(X)$ endlich oder abz�hlbar ist.

\begin{definition}[Gewichtsfunktion]
Die \emph{Wahrscheinlichkeitsfunktion} oder \emph{Gewichtsfunktion} einer diskreten Zufallsvariable $X$ ist gegeben durch
$$
p_X(x) =
\left\{\begin{array}{ll}
\P[X = x] & \text{f�r }x \in \mathcal{W}(X) \\
0         & \text{sonst}
\end{array}\right.
$$
\end{definition}

Eine Gewichtsfunktion weist folgende Eigenschaften auf:
\begin{compactenum}[i:]
\item $p_X(x) \in [0,1]$ f�r alle $x$.
\item $\sum_{x_i \in \mathcal{W}(X)} p_X(x_i) = 1$.
\end{compactenum}

\begin{definition}[Diskrete Verteilungsfunktion]
Die Verteilungsfunktion $F_X$ einer diskreten Zufallsvariable $X$ mit Wertebereich $\mathcal{W}(X) = \{x_1,\ldots,x_n\}$ ist die Funktion
$$
F_X(t) =
\P[X \leq t] =
\sum_{x_k \in \mathcal{W}(X) \atop x_k \leq t} p_X(x_k)
$$
\end{definition}


% ------------------------------------------------------------------------------------------------ %
% Stetige zufallsvariablen
% ------------------------------------------------------------------------------------------------ %


\subsection{Stetige Zufallsvariablen}

\begin{definition}[Dichte]
Eine Zufallsvariable $X$ mit der Verteilungsfunktion $F_X(t)$ heisst stetig mit \emph{Dichte} $f_X:\R\rightarrow[0,\infty)$, falls gilt
$$
F_X(t) = \int_{-\infty}^t f_X(s) \d s
\quad
\text{f�r alle } t \in \R.
$$
\end{definition}

F�r eine Dichtefunktion $f_X$ gilt:
\begin{compactenum}[i:]
\item $f_X(t) \geq 0$ f�r alle $t \in \R$.
\item $\int_{-\infty}^\infty f_X(s) \d s = 1$.
\end{compactenum}

\begin{note}
Es gilt $\frac{\d}{\d t} F_X(t) = f_X(t)$ falls $f_X$ an der Stelle $t$ stetig ist.
\end{note}


% ------------------------------------------------------------------------------------------------ %
% TRANSFORMATION
% ------------------------------------------------------------------------------------------------ %


\subsection{Transformation von Zufallsvariablen}

\begin{theorem}
Sei $X$ eine stetige Zufallsvariable mit Dichte $f_X$ und $f_X(t) = 0$ f�r $t \notin I \subseteq \R$. Sei $g:\R\rightarrow\R$ stetig differenzierbar und streng monoton auf $I$ mit Umkehrfunktion $g^{-1}$. Dann hat die Zufallsvariable $Y := g(X)$ die Dichte
$$
f_Y =
\left\{\begin{array}{ll}
f_X (g^{-1}(t)) \lvert\frac{\d}{\d t} g^{-1}(t)\rvert & \text{f�r } t \in \{g(x) \mid x \in I\} \\[1ex]
0 & \text{sont}
\end{array}\right.
$$
\end{theorem}

\begin{example}[Lineare Transformation] Aus $Y := aX+b$ mit $a>0,b \in\R$ folgt
$$
F_Y(t) =
\P[aX+b \leq t] =
\P\left[X \leq \frac{t-b}{a}\right] =
F_X\left(\frac{t-b}{a}\right)
$$
und mit der Kettenregel ergibt sich
$$
f_Y(t) = \frac{\d}{\d t}F_Y(t) = \frac{1}{a} f_X\left(\frac{t-b}{a}\right).
$$
\end{example}

\begin{example}[Nichtlineare Transformation] Aus $Y := X^2$ folgt
$$
F_Y(t) =
\P[X^2 \leq t] =
\P\bigl[-\sqrt{t} \leq X \leq \sqrt{t}\bigr] =
F_X\bigl(\sqrt{t}\bigr)-F_X\bigl(-\sqrt{t}\bigr)
$$
und somit
$$
f_Y(t) = \frac{\d}{\d t}F_Y(t) =
\frac{f_X\bigl(\sqrt{t}\bigr)+f_X\bigl(-\sqrt{t}\bigr)}{2\sqrt{t}}
$$
\end{example}


% ------------------------------------------------------------------------------------------------ %
% SIMULATION VON VERTEILUNGEN
% ------------------------------------------------------------------------------------------------ %

\subsection{Simulation von Verteilungen}

\begin{theorem}
Sei $F$ eine stetige und streng monoton wachsende Verteilungsfunktion mit Umkehrfunktion $F^{-1}$. Ist $X \sim \mathcal{U}(0,1)$ und $Y := F^{-1}(X)$, so hat $Y$ die Verteilungsfunktion $F$.
\end{theorem}

\begin{example} Um die Verteilung $Exp(\lambda)$ zu simulieren bestimmt man zu der Verteilungsfunktion $F(t) = 1-e^{-\lambda t}$ f�r $t \geq 0$ die Inverse $F^{-1}(t) = -\frac{\log(1-t)}{\lambda}$. Mit $U \sim \mathcal{U}(0,1)$ erh�lt man
$$
X := F^{-1}(U) = -\frac{\log(1-U)}{\lambda} \sim Exp(\lambda).
$$
\end{example}

% ------------------------------------------------------------------------------------------------ %