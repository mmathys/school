% ------------------------------------------------------------------------------------------------ %


\section{Wichtige Verteilungen}


% ------------------------------------------------------------------------------------------------ %
% DISKRETE VERTEILUNGEN
% ------------------------------------------------------------------------------------------------ %


\subsection{Diskrete Verteilungen}


% ------------------------------------------------------------------------------------------------ %
% DISKRETE GLEICHVERTEILUNG
% ------------------------------------------------------------------------------------------------ %


\subsubsection{Diskrete Gleichverteilung}

Zufallsvariable $X$ mit Wertebereich $\mathcal{W}(X) = \{x_1,\ldots,x_n\}$ und alle Werte haben die gleiche Wahrscheinlichkeit falls
$$
p_X(x_i) = \frac{1}{n}
\quad
\text{f�r } i \in \{1,\ldots,n\}
$$

\begin{example}[W�rfeln] Die Zufallsvariable $X$ gibt die Augenzahl bei einem W�rfelwurf an. Der Wertebereich ist also
$\mathcal{W} = \{1,2,3,4,5,6\}$ und somit $n=6$.
\end{example}


% ------------------------------------------------------------------------------------------------ %
% BERNOULLI VERTEILUNG
% ------------------------------------------------------------------------------------------------ %


\subsubsection{Bernoulli-Verteilung}

Eine bernoulli-verteilte Zufallsvariable $X \sim Be(p)$ mit Parameter $p \in [0,1]$ nimmt die Werte $0$ und $1$ mit Wahrscheinlichkeiten
$$
p_X(1) = p
\quad\text{und}\quad
p_X(0) = 1-p
$$
an. Eine alternative Schreibweise ist
$$
p_X(x) =
\left\{\begin{array}{ll}
p^x (1-p)^{1-x} & x \in \{0,1\} \\
0 & \text{sonst}.
\end{array}\right.
$$

\begin{highlight}
\begin{tabular}{l@{ : }l}
Erwartungswert & $p$ \\
Varianz        & $p(1-p)$
\end{tabular}
\end{highlight}

\begin{example}[M�nzwurf] Ein fairer M�nzwurf ist bernoulli-verteilt mit Parameter $p=\frac{1}{2}$. F�r einen Parameter $p \neq \frac{1}{2}$ w�re der M�nzwurf unfair.
\end{example}


% ------------------------------------------------------------------------------------------------ %
% BINOMIALVERTEILUNG
% ------------------------------------------------------------------------------------------------ %


\subsubsection{Binomialverteilung}

Die Gewichtsfunktion $p_X$ einer binomial-verteilten Zufallsvariable $X \sim Bin(n,p)$ mit Parameter $n \in \N$ und $p \in [0,1]$ ist gegeben durch
$$
p_x(k) = {n \choose k} p^k (1-p)^{n-k}
\quad
\text{f�r } k \in \{0,\ldots,n\}
$$

\begin{highlight}
\begin{tabular}{l@{ : }l}
Erwartungswert & $np$ \\
Varianz        & $np(1-p)$
\end{tabular}
\end{highlight}

$X$ ist die Anzahl der Erfolge $k$ bei $n$ unabh�ngigen Wiederholungen eines Bernoulli-Experiments. Es gibt ${n \choose k}$ verschiedene M�glichkeiten bei $n$ Versuchen $k$-mal erfolgreich zu sein. Jeder dieser M�glichkeit hat Wahrscheinlichkeit $p^k (1-p)^{n-k}$.


% ------------------------------------------------------------------------------------------------ %
% GEOMETRISCHE VERTEILUNG
% ------------------------------------------------------------------------------------------------ %


\subsubsection{Geometrische Verteilung}

Die Gewichtsfunktion $p_X$ einer geometrisch-gleichverteilten Zufallsvariable $X \sim Geom(p)$  mit Parameter $p \in [0,1]$ ist gegeben durch
$$
p_X(k) = p(1-p)^{k-1}
\quad
\text{f�r } k \in \{1,2,\ldots\}
$$

\begin{highlight}
\begin{tabular}{l@{ : }l}
Erwartungswert & $\frac{1}{p}$ \\
Varianz        & $\frac{1}{p^2}(1-p)$
\end{tabular}
\end{highlight}

\begin{example}[Wartezeit] Die Geometrische Verteilung ist die Wahrscheinlichkeitsverteilung der Anzahl $X$ Bernoulli-Versuche, die notwendig sind, um den ersten Erfolg zu erzielen. F�r die Anzahl W�rfelfw�rfe, die man braucht um eine $6$ zu w�rfeln, ist $p = \frac{1}{6}$.
\end{example}


% ------------------------------------------------------------------------------------------------ %
% NEGATIVBINOMIALE VERTEILUNG
% ------------------------------------------------------------------------------------------------ %


\subsubsection{Negativbinomiale Verteilung}

Die Gewichtsfunktion $p_X$ einer negativ-binomial-verteilten Zufallsvariable $X$ mit Parameter $r \in \N$ und $p \in [0,1]$ ist gegeben durch
$$
p_X(k) = {{k-1} \choose {r-1}}p^r(1-p)^{k-r}
\quad
\text{f�r } k \in \{r,r+1,\ldots\}
$$

\begin{highlight}
\begin{tabular}{l@{ : }l}
Erwartungswert & $\frac{r}{p}$ \\
Varianz        & $\frac{r}{p^2}(1-p)$
\end{tabular}
\end{highlight}

$X$ entspricht der Wartezeit auf den $r$-ten Erfolg. Es gibt ${{k-1} \choose {r-1}}$ m�glichkeiten f�r $r-1$ Erfolge bei $k-1$ Versuchen; der $r$-te Erfolg tritt ja beim $k$-ten Versuch ein.


% ------------------------------------------------------------------------------------------------ %
% HYPERGEOMETRISCHE VERTEILUNG
% ------------------------------------------------------------------------------------------------ %


\subsubsection{Hypergeometrische Verteilung}

Die Gewichtsfunktion $p_X$ einer hypergeometrisch-verteilten Zufallsvariable $X$ mit Parameter $r,n,m \in \N$, wobei $r,m \leq n$, ist gegeben durch
$$
p_X(k) = \frac{{r \choose k}{{n-r} \choose {m-k}}}{{n \choose m}}
\quad
\text{f�r } k \in \{0,\ldots,\min\{r,m\}\}
$$

\begin{highlight}
\begin{tabular}{l@{ : }l}
Erwartungswert & $m\frac{r}{n}$ \\
Varianz        & $m\frac{r}{n}(1-\frac{r}{n})\frac{n-m}{n-1}$
\end{tabular}
\end{highlight}

In einer Urne befinden sich $n$ Gegenst�nde. Davon sind $r$ Gegenst�nde vom Typ A und $n-r$ vom Typ B. Es werden $m$ Gegenst�nde ohne Zur�cklegen gezogen. $X$ beschreibt die Wahrscheinlichkeitsverteilung f�r die Anzahl $k$ der Gegenst�nde vom Typ A in der Stichprobe.

\begin{example}[Lotto]
Anzahl Zahlen $n=45$, richtige Zahlen $r=6$, meine Zahlen $m=6$. Die Wahrscheinlichkeit f�r $4$ Richtige ist
$$
p_X(4) = \frac{{6 \choose 4}{39 \choose 2}}{{45 \choose 6}} \approx 0.00136.
$$
\end{example}


% ------------------------------------------------------------------------------------------------ %
% POISSONVERTEILUNG
% ------------------------------------------------------------------------------------------------ %


\subsubsection{Poisson Verteilung}

Die Gewichtsfunktion $p_X$ einer Poisson-verteilten Zufallsvariable $X \sim \mathcal{P}(\lambda)$ mit Parameter $\lambda$ ist gegeben durch
$$
p_X(k) = e^{-\lambda}\frac{\lambda^k}{k!}
\quad
\text{f�r } k \in \{0,1,\ldots\}
$$

\begin{highlight}
\begin{tabular}{l@{ : }l}
Erwartungswert & $\lambda$ \\
Varianz        & $\lambda$
\end{tabular}
\end{highlight}

Die Poisson-Verteilung eignet sich zur Modellierung von seltenen Ereignissen, wie z.B. Versicherungssch�den.

\begin{note}
Die Binomialverteilung $Bin(n,p)$ kann approximativ durch die Poissonverteilung $\mathcal{P}(\lambda)$ mit $\lambda = np$ berechnet werden.
Faustregel: Die Approximation kann f�r $np^2 \leq 0.05$ benutzt werden.
\end{note}


% ------------------------------------------------------------------------------------------------ %
% STETIGE VERTEILUNGEN
% ------------------------------------------------------------------------------------------------ %


\subsection{Stetige Verteilungen}


% ------------------------------------------------------------------------------------------------ %
% STETIGE GLEICHVERTEILUNG
% ------------------------------------------------------------------------------------------------ %


\subsubsection{Stetige Gleichverteilung}
Die Dichte $f_X$ und Verteilungsfunktion $F_X$ einer stetigen und gleichverteilten Zufallsvariable $X \sim \mathcal{U}(a,b)$ mit Parameter $a,b \in \R$ wobei $a < b$ sind gegeben durch
$$
\begin{array}{rcl}
f_X(t) & = &
\left\{\begin{array}{ll}
\frac{1}{b-a} & \text{f�r } t \in [a,b] \\[1ex]
0             & \text{f�r } t \notin [a,b]
\end{array}\right.
\\[1.5em]
F_X & = &
\left\{\begin{array}{ll}
0               & \text{f�r } t < a \\[1ex]
\frac{t-a}{b-a} & \text{f�r } t \in [a,b] \\[1ex]
1               & \text{f�r } t > b
\end{array}\right.
\end{array}
$$

\begin{highlight}
\begin{tabular}{l@{ : }l}
Erwartungswert & $\frac{1}{2}(a+b)$ \\
Varianz        & $\frac{1}{12}(a-b)^2$
\end{tabular}
\end{highlight}

\begin{example}
Rundungsfehler einer Messung.
\end{example}

% ------------------------------------------------------------------------------------------------ %
% EXPONENTIALVERTEILUNG
% ------------------------------------------------------------------------------------------------ %


\subsubsection{Exponentialverteilung}

Die Dichte $f_X$ und Verteilungsfunktion $F_X$ einer exponential-verteilten Zufallsvariable $X \sim Exp(\lambda)$ mit Parameter $\lambda > 0$ sind gegeben durch
$$
\begin{array}{rcl}
f_X(t) & = & 
\left\{\begin{array}{ll}
\lambda e^{-\lambda t} & \text{f�r } t \geq 0 \\[1ex]
0                      & \text{f�r } t < 0
\end{array}\right.
\\[1.5em]
F_X(t) & = &
\left\{\begin{array}{ll}
1 - e^{-\lambda t} & \text{f�r } t \geq 0 \\[1ex]
0                  & \text{f�r } t < 0
\end{array}\right.
\end{array}
$$

\begin{highlight}
\begin{tabular}{l@{ : }l}
Erwartungswert & $\frac{1}{\lambda}$ \\
Varianz        & $\frac{1}{\lambda^2}$
\end{tabular}
\end{highlight}

\begin{example}[Lebensdauer]
Die Exponentialverteilung ist eine typische Lebensdauerverteilung. So ist beispielsweise die Lebensdauer von elektronischen Bauelementen h�ufig ann�hernd exponentialverteilt.
\end{example}


% ------------------------------------------------------------------------------------------------ %
% Normalverteilung
% ------------------------------------------------------------------------------------------------ %


\subsubsection{Normalverteilung}

Die Dichte $f_X$ einer normalverteilten Zufallsvariable $X \sim \mathcal{N}(\mu,\sigma^2)$ mit Parameter $\mu \in \R$ und $\sigma^2 > 0$ ist gegeben durch
$$
f_X(t) =
\frac{1}{\sigma\sqrt{2\pi}} \, e^{-\frac{(t-\mu)^2}{2\sigma^2}}
$$
F�r die Verteilungsfunktion $F_X$ existiert kein geschlossener Ausdruck. Deshalb werden die Werte der Verteilungsfunktion $\Phi(t)$ der \emph{Standard-Normalverteilung}
$$
f_X(t) =
\frac{1}{\sqrt{2\pi}} \, e^{-\frac{t^2}{2}}
$$
mit $\mu=0$ und $\sigma^2=1$ tabelliert. F�r allgemeine Normalverteilungen berechnet man dann
$$
F_X(t) = \P[X \leq t] = \P\left[\frac{X-\mu}{\sigma}\leq\frac{t-\mu}{\sigma}\right] = \Phi\left(\frac{t-\mu}{\sigma}\right).
$$

\begin{highlight}
\begin{tabular}{l@{ : }l}
Erwartungswert & $\mu$ \\
Varianz        & $\sigma^2$
\end{tabular}
\end{highlight}

\begin{example}
Streuung von Messwerten um den Mittelwert.
\end{example}


% ------------------------------------------------------------------------------------------------ %