\section{Kovalente Bindung}

\subsection{Strukturschreibweisen}

\begin{itemize}
	\item Strukturformel
	\item Skelettformel = Lewis-Formel
	\item Gruppenformel
\end{itemize}

\subsection{Struktur und Geometrie von Molekülen, Elektronennegativität, Polarität}

\subsubsection{Orbitale}

\begin{itemize}
	\item s: 2 Elektronen
	\item p: 6 Elektronen
	\item d: 10 Elektronen
	\item f: 14 Elektronen
\end{itemize}

\subsubsection{Elektronenkonfiguration Schreibweise}

C: $1s^2 \ 2s^2 \ 2p^2$ oder $[He] \ 2s^2 \ 2p^2$

\subsubsection{Elektronennegativität}

\begin{definition}[Elektronennegativität]
	$\chi$; relatives Mass für die Fähigkeit eines Atoms, in einer chemischen Bindung Elektronenpaare an sich zu ziehen. Von 0.7 bis 4.
\end{definition}

Bei $\Delta_\chi \geqslant 1.8$: Ionenbindung.

\subsection{Zwischenmolekulare Kräfte}

\paragraph{Van der Waals-Kräfte.}

TODO

\paragraph{Dipol-Dipol-Wechselwirkung.}

TODO

\paragraph{Wasserstoffbrücken.} Starke Dipol-Dipol-Wechselwirkungen. TODO

\subsection{Eigenschaften molekularer Stoffe}

\paragraph{Schmelzpunkt, Siedepunkt.}

Eher tief, viele Molekülverbindungen sind bei Raumtemperatur flüssig oder gasförmig. Abhängig von zwischenmolekularen Kräften (ZMK).

\paragraph{Löslichkeit.}

Abhängig von ZMK. Polare Moleküle wasserlöslich, unpolare löslich in unpolaren Lösungsmitteln (Benzin).

\paragraph{Sonstiges.}

Elektrische Nichtleiter.


