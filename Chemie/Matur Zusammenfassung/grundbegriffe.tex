% ------------------------------------------------------------------------------------------------ %
% GRUNDLAGEN
% ------------------------------------------------------------------------------------------------ %

\section{Wahrscheinlichkeiten}


% ------------------------------------------------------------------------------------------------ %
% Ereignissraum
% ------------------------------------------------------------------------------------------------ %


\subsection{Ereignisraum, Grundraum}

\begin{definition}[Ereignisraum]
Der \emph{Ereignisraum} oder \emph{Grundraum} $\Omega$ ist die Menge aller m�glichen Ereignisse eines Zufallexperiments. Die Elemente $\omega \in \Omega$ heissen \emph{Elementarereignisse}.
\end{definition}

\begin{definition}[Ereignis]
Ein \emph{Ereignis} $A \subseteq \Omega$ ist eine Teilmenge von $\Omega$.
\end{definition}

Die Klasse aller \emph{beobachtbaren Ereignisse} $\mathcal{F}$ ist eine Teilmenge der Potenzmenge $\mathcal{P}(\Omega)$ von $\Omega$.


% ------------------------------------------------------------------------------------------------ %
% Das Wahrscheinlichkeitsmass
% ------------------------------------------------------------------------------------------------ %


\subsection{Wahrscheinlichkeitsmass}

\begin{definition}[Wahrscheinlichkeitsmass]
Ein \emph{Wahrscheinlichkeitsmass} $\P$ ist eine Abbildung $\P:\mathcal{F} \rightarrow [0,1]$ mit folgenden Eigenschaften:
\begin{compactenum}[i:]
\item $\P[\Omega] = 1$.
\item $\P[A] \geq 0$ f�r alle $A \in \mathcal{F}$.
\item $\P[\bigcup_{i=1}^\infty A_i] = \sum_{i=1}^\infty \P[A_i]$ falls $A_i \cap A_j = \emptyset$ f�r $i \neq j$.
\end{compactenum}
\end{definition}

Aus den Axiomen i bis iii folgen direkt die Rechenregeln:
\begin{compactenum}[i:]
\item $\P[A^C] = 1 - \P[A]$.
\item $\P[\emptyset] = 0$.
\item $A \subseteq B \Rightarrow \P[A] \leq \P[B]$.
\item $\P[A \cup B] = \P[A]+\P[B]-\P[A \cap B]$ (\emph{Additionsregel}).
\end{compactenum}


% ------------------------------------------------------------------------------------------------ %

\subsection{Endliche R�ume}

F�r einen endlichen Raum $\Omega = \{\omega_1,\ldots,\omega_n\}$ mit $\P[\omega_i] = p_i$ f�r alle $1 \leq i \leq n$ gilt
$$
\P[A] = \sum_{ i \text{ mit } w_i \in A} p_i
$$


\begin{definition}[Laplace-Raum]
In einem \emph{Laplace-Raum} sind alle Ereignisse $\omega_1,\ldots,\omega_n$ gleich wahrscheinlich ($p_i = p_j$ f�r alle $i,j$). Es gilt dann
$$
P[A] = \frac{\lvert A \rvert}{\lvert \Omega \rvert}
$$
\end{definition}


% ------------------------------------------------------------------------------------------------ %
% BEDINGTE WAHRSCHEINLICHKEIT, TOTALE WAHRSCHEINLICHKEIT UND FORMEL VON BAYES
% ------------------------------------------------------------------------------------------------ %


\subsection{Bedingte Wahrscheinlichkeit}

\begin{definition}[Bedingte Wahrscheinlichkeit]
Seien $A,B$ Ereignisse und $\P[A] > 0$, dann ist die \emph{bedingte Wahrscheinlichkeit} von $B$ gegeben $A$ definiert durch:
$$
\P[B \mid A] := \frac{\P[A \cap B]}{\P[A]}
$$
\end{definition}

Aus der Definition der bedingten Wahrscheinlichkeit folgt die \emph{Multiplikationsregel}:
$$
\P[A \cap B] = \P[B \mid A] \P[A]
$$

\begin{theorem}[Totale Wahrscheinlichkeit]
Sei $A_{1 \leq i \leq n}$ eine disjunkte Zerlegung von $\Omega$, dann gilt f�r ein beliebiges Ereignis $B$:
$$
\P[B] = \sum_{i=1}^n \P[B \mid A_i]\P[A_i]
$$
\end{theorem}

\begin{theorem}[Formel von Bayes]
Sei $A_1,\ldots,A_n$ eine disjunkte Zerlegung von $\Omega$ mit $\P[A_i] > 0$ f�r alle $i$ und $B$ ein Ereignis mit $\P[B] > 0$, dann gilt f�r jedes $k$:
$$
\P[A_k \mid B] = \frac{\P[B \mid A_k] \P[A_k]}{\sum_{i=1}^n \P[B \mid A_i] \P[A_i]}
$$
\end{theorem}


% ------------------------------------------------------------------------------------------------ %
% UNABH�NGIGKEIT
% ------------------------------------------------------------------------------------------------ %


\subsection{Unabh�ngigkeit}

\begin{definition}[Unabh�ngigkeit]
Die Ereignisse $A_1,\ldots,A_n$ heissen \emph{unabh�ngig}, wenn f�r alle $m \in \N$ und $\{k_1,\ldots,k_m\} \subseteq \{1,\ldots,n\}$ gilt

$$
\P\left[\bigcap_{i=1}^m A_{k_i}\right] = \prod_{i=1}^m \P[A_{k_i}].
$$
\end{definition}

\begin{note}
Bei unabh�ngigen Ereignissen $A,B$ hat das Eintreten des einen Ereignisses keinen Einfluss auf die Wahrscheinlichkeit des anderen Ereignisses: $\P[B \mid A] = \frac{\P[A \cap B]}{\P[A]} = \P[B]$
\end{note}


% ------------------------------------------------------------------------------------------------ %