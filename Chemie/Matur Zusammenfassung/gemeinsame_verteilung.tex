% ------------------------------------------------------------------------------------------------ %
% GEMEINSAME UND BEDINGTE WAHRSCHEINLICHKEITEN
% ------------------------------------------------------------------------------------------------ %


\section{Gemeinsame Verteilungen}

\begin{definition}[Gemeinsame Verteilung]
Die \emph{gemeinsame Verteilungsfunktion} von $n$ Zufallsvariablen $X_1,\ldots,X_n$ ist die Abbildung $F:\R^n \rightarrow [0,1]$,
$$
F(t_1,\ldots,t_n) := \P[X_1 \leq t_1,\ldots,X_n \leq t_n].
$$
\end{definition}

\begin{definition}[Gemeinsame Gewichtsfunktion]
Falls $X_1,\ldots,X_n$ diskrete Zufallsvariablen sind, ist ihre \emph{gemeinsame Gewichtungsfunktion} $p:\R^n\rightarrow [0,1]$ definiert durch
$$
p(x_1,\ldots,x_n) := \P[X_1=x_1,\ldots,X_n=x_n].
$$
\end{definition}

\begin{definition}[Gemeinsame Dichte]
Seien $X_1,\ldots,X_n$ stetige Zufallsvariablen mit gemeinsamer Verteilungsfunktion $F(t_1,\ldots,t_n)$. Die Funktion $f : \R^n \rightarrow [0,\infty)$ heisst \emph{gemeinsame Dichte} von $X_1,\ldots,X_n$, falls f�r alle $t_i \in \R$ gilt
$$
F(t_1,\ldots,t_n) = \int_{-\infty}^{t_1} \ldots \int_{-\infty}^{t_n} f(x_1,\ldots,x_n) \d x_n \ldots \d x_1
$$
\end{definition}

Falls $X_1,\ldots,X_n$ eine gemeinsame Dichte $f$ haben, so hat diese folgende Eigenschaften:
\begin{compactenum}[i:]
\item $f(t_1,\ldots,t_n) \geq 0$ f�r alle $t_i \in \R$.
\item $\int_{\R^n} f(t_1,\ldots,t_n) \d\mu = 1$.
\item $\P[(X_1,\ldots,X_n) \in A] = \int_{(t_1,\ldots,t_n)\in A} f(t_1,\ldots,t_n) \d\mu$.
\item $f(t_1,\ldots,t_n) = \frac{\partial^n}{\partial t_1 \cdots \partial t_n} F(t_1,\ldots,t_n)$, falls definiert.
\end{compactenum}


% ------------------------------------------------------------------------------------------------ %
% RANDVERTEILUNG
% ------------------------------------------------------------------------------------------------ %


\subsection{Randverteilungen}

\begin{definition}[Randverteilung]
Seien $X$ und $Y$ Zufallsvariablen mit gemeinsamer Verteilungsfunktion $F_{X,Y}$, dann ist die \emph{Randverteilung} $F_X:\R \rightarrow [0,1]$ von $X$ definiert durch
$$
F_X := \P[X \leq x] = \P[X \leq x, Y < \infty] = \lim_{y\rightarrow\infty}F_{X,Y}(x,y).
$$
\end{definition}

F�r zwei diskrete Zufallsvariablen $X$ und $Y$ mit gemeinsamer Gewichtsfunktion $p_{X,Y}(x,y)$ ist die Gewichtsfunktion der Randverteilung von $X$ gegeben durch
$$
p_X =
\P[X=x] =
\sum_j \P[X=x,Y=y_j] =
\sum_j p_{X,Y}(x,y_j).
$$

F�r zwei stetige Zufallsvariablen $X$ und $Y$ mit gemeinsamer Dichte $f_{X,Y}(x,y)$ ist die Randdichte (Dichtefunktion der Randverteilung) von $X$ gegeben durch
$$
f_X(x) = \int_{-\infty}^\infty f_{X,Y}(x,y) \d y.
$$


% ------------------------------------------------------------------------------------------------ %
% BEDINGTE VERTEILUNG
% ------------------------------------------------------------------------------------------------ %


\subsection{Bedingte Verteilung}

\begin{definition}[Bedingte Gewichtsfunktion]
Seien $X$ und $Y$ diskrete Zufallsvariablen mit gemeinsamer Gewichtsfunktion $p_{X,Y}(x,y)$, dann ist die \emph{bedingte Gewichtsfunktion} $p_{X \mid Y}(x \mid y)$ von $X$ gegeben $Y$ definiert durch
$$
p_{X \mid Y}(x \mid y) := \P[X = x \mid Y = y] = \frac{p_{X,Y}(x,y)}{p_Y(y)}
$$
falls $p_Y(y) > 0$ und $0$ falls $p_Y(y) = 0$.
\end{definition}

\begin{definition}[Bedingte Dichte]
F�r zwei stetige Zufallsvariablen $X$ und $Y$ mit gemeinsamer Dichte $f_{X,Y}(x,y)$ ist die \emph{bedingte Dichte} $f_{X \mid Y}$ von $X$ gegeben $Y$ definiert durch
$$
f_{X \mid Y}(x \mid y) := \frac{f_{X,Y}(x,y)}{f_Y(y)}
$$
falls $f_Y(y) > 0$ und $0$ falls $f_Y(y) = 0$.
\end{definition}


% ------------------------------------------------------------------------------------------------ %
% UNABH�NGIGKEIT
% ------------------------------------------------------------------------------------------------ %


\subsection{Unabh�ngigkeit}

\begin{definition}[Unabh�ngigkeit]
Die Zufallsvariablen $X_1,\ldots,X_n$ heissen \emph{unabh�ngig}, falls die gemeinsame Verteilungsfunktion das Produkt der Verteilungsfunktionen der Randverteilungen ist:
$$
F(x_1,\ldots,x_n) = \prod_{i=1}^n F_{X_i}(x_i)
$$
\end{definition}

Im diskreten Fall sind $X_1,\ldots,X_n$ unabh�ngig, genau dann wenn
$$
p(x_1,\ldots,x_n) = \prod_{i=1}^n p_{X_i}(x_i)
$$
gilt und analog im stetigen Fall, falls
$$
f(x_1,\ldots,x_n) = \prod_{i=1}^n f_{X_i}(x_i).
$$


% ------------------------------------------------------------------------------------------------ %