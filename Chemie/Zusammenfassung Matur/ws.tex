%% ------------------------------------------------------------------------------------------------ %
% ZUSAMMENFASSUNG
% WAHRSCHEINLICHKEIT UND STATISTIK
% ------------------------------------------------------------------------------------------------ %
% Autor: J�r�me Dohrau
% Email: dohrauj @ ...
% Die Zusammenfassung darf gerne f�r den eigenen Gebrauch angepasst werden. Falls jemand einen
% Fehler findet, w�rde ich mich �ber eine Benachrichtigung per Email freuen.
% ------------------------------------------------------------------------------------------------ %


\documentclass[a4paper,twocolumn]{article}


% ------------------------------------------------------------------------------------------------ %
% ALLGEMEINE PAKETE
% ------------------------------------------------------------------------------------------------ %


% Silbentrennung, Sonderzeichen ect.
\usepackage[german]{babel}
\usepackage[T1]{fontenc}


%\usepackage[utf8]{inputenc} do not use bc emoji
% emoji
\usepackage{coloremoji}


% Mathematische Zeichen
\usepackage{amssymb}
% Lehrs�tze
\usepackage{amsthm}
% Mathematische Extras (Matrizen ect.)
\usepackage{amsxtra}
\usepackage{gensymb}

% F�r kompakte Listen
\usepackage{paralist}

% Rahmen
\usepackage{framed}



% Grafik
%\usepackage{graphicx}
\usepackage{ifpdf}
\usepackage{psfrag}
\usepackage{epstopdf}

\usepackage{titlesec}
\usepackage{titletoc}

% Graphs and Automata
\usepackage{gastex}

% farben
\usepackage{color}
\usepackage{bbold}


%chemie
\usepackage{chemfig}


% ------------------------------------------------------------------------------------------------ %
% FARBEN
% ------------------------------------------------------------------------------------------------ %


\definecolor{headtext}{rgb}{0.50,0.50,0.50}
\definecolor{foottext}{rgb}{0.50,0.50,0.50}
\definecolor{headsepline}{rgb}{0.80,0.80,0.80}
\definecolor{footsepline}{rgb}{0.80,0.80,0.80}


% Schattierung f�r Hinweisboxen
\definecolor{shadecolor}{rgb}{0.88,0.91,0.95}

\definecolor{highlightcolor}{rgb}{1.00,0.50,0.50}


% ------------------------------------------------------------------------------------------------ %
% SEITEN- UND TEXTLAYOUT
% ------------------------------------------------------------------------------------------------ %


% Seitenr�nder
\usepackage[left=10mm,right=10mm,top=20mm,bottom=20mm]{geometry}

% Zeileneinzug am Anfang eines Absatzes
\setlength{\parindent}{0em}


% ------------------------------------------------------------------------------------------------ %
% KOPF- UND FUSSZEILEN
% ------------------------------------------------------------------------------------------------ %


\usepackage{fancyhdr}
\pagestyle{fancy}

\fancyhead{} % l�schen
\fancyhead[L]{\color{headtext}\uppercase{Chemie Matur Zusammenfassung}}
\fancyhead[R]{\color{headtext}\leftmark}

\fancyfoot{} % l�schen
\fancyfoot[L]{\color{foottext}\uppercase{Seite} \thepage}
\fancyfoot[R]{\color{foottext}\uppercase{Max Mathys}}

\renewcommand{\headrulewidth}{1pt}
\renewcommand{\headrule}{{\color{headsepline}\hrule width\headwidth height\headrulewidth \vskip-\headrulewidth}}
\renewcommand{\footrulewidth}{1pt}
\renewcommand{\footrule}{{\color{footsepline}\vskip-\footruleskip\vskip-\footrulewidth\hrule width\headwidth height\footrulewidth\vskip\footruleskip}}


% ------------------------------------------------------------------------------------------------ %
% LEHRS�TZE (DEFINITIONEN ETC.)
% ------------------------------------------------------------------------------------------------ %


\newtheoremstyle{defstyle}
	% Abstand oben
	{10pt}
	% Abstand unten
	{10pt}
	% Schriftart
	{\normalfont}
	% Zeileneinzug (leer = kein Zeileneinzug, \parindent = Absatz-Zeileneinzug)
	{}
	% Titel Schriftart
	{\normalfont\bfseries}
	% Zeichen nach dem Titel
	{:} 
	% Leerraum nach dem Titel
	{ }
	% Lehrsatzkopf definieren (leer = normal)
	{#3}

\newtheoremstyle{thmstyle}
	% Abstand oben
	{10pt}
	% Abstand unten
	{10pt}
	% Schriftart
	{\normalfont}
	% Zeileneinzug (leer = kein Zeileneinzug, \parindent = Absatz-Zeileneinzug)
	{}
	% Titel Schriftart
	{\normalfont\bfseries}
	% Zeichen nach dem Titel
	{:} 
	% Leerraum nach dem Titel
	{ }
	% Lehrsatzkopf definieren (leer = normal)
	{\thmname{#1}\thmnumber{ #2}\thmnote{ (#3)}}

\theoremstyle{defstyle}
% Definition
\newtheorem*{definition}{Definition}

\theoremstyle{thmstyle}
% 
\newtheorem*{theorem}{Satz}
% Beachte
\newtheorem*{tnote}{Hinweis}

% Beispiel
\newtheorem*{example}{Beispiel}

\newenvironment{note}
	{\begin{snugshade}\begin{tnote}}
	{\end{tnote}\end{snugshade}}
	
\newenvironment{highlight}
	{\begin{snugshade}}
	{\end{snugshade}}


% ------------------------------------------------------------------------------------------------ %
% ABBILDUNGEN
% ------------------------------------------------------------------------------------------------ %


\usepackage[bf]{caption}

\addto\captionsgerman{
    \renewcommand{\figurename}{Abb.}
    \renewcommand{\tablename}{Tab.}}

% Abbildungs�berschrift
\renewcommand{\captionfont}{\footnotesize\sffamily}


% ------------------------------------------------------------------------------------------------ %
% SELBSTDEFINIERTE BEFEHLE
% ------------------------------------------------------------------------------------------------ %


\newcommand{\todo}[1]{{\definecolor{shadecolor}{rgb}{1.00,0.30,0.30}\begin{snugshade}{\bf TODO:} #1\end{snugshade}}}

\newcommand{\R}{\mathbb{R}}
\newcommand{\N}{\mathbb{N}}
\renewcommand{\P}{\mathbb{P}}
\newcommand{\E}{\mathbb{E}}

\renewcommand{\d}{\mathrm{d}}

\renewcommand{\theta}{\vartheta}
\renewcommand{\phi}{\varphi}

\newcommand{\var}{\mathrm{Var}}
\newcommand{\cov}{\mathrm{Cov}}
\newcommand{\sd}{\mathrm{sd}}
\newcommand{\corr}{\mathrm{Corr}}

\newcommand{\abs}[1]{\lvert #1 \rvert}


% ------------------------------------------------------------------------------------------------ %
% LAYOUT INHALTSVERZEICHNIS UNDSO
% ------------------------------------------------------------------------------------------------ %


\renewcommand{\thepart}{\Alph{part}}
\renewcommand{\thesection}{\Alph{part}.\arabic{section}}

\titlecontents{part}[0em]%
{\vspace{1em}\bf\large}{\thecontentslabel\enskip}{}%
{\hfill\contentspage}

\titlecontents{section}[0em]%
{\vspace{0.3em}\bf}{\thecontentslabel\enskip}{}%
{\hfill\contentspage}

\titlecontents{subsection}[1em]%
{}{\thecontentslabel\enskip}{}%
{\titlerule*[0.6em]{.}\contentspage}

\titlecontents{subsubsection}[2em]%
{}{\thecontentslabel\enskip}{}%
{\titlerule*[0.6em]{.}\contentspage}


% ------------------------------------------------------------------------------------------------ %
% INHALT
% ------------------------------------------------------------------------------------------------ %


\begin{document}

\setcounter{tocdepth}{2}
\tableofcontents

\hfill\newpage

\setcounter{part}{1}
\setcounter{section}{0}
\part*{Atomlehre}
\addcontentsline{toc}{part}{Atomlehre}

\section{Atommodelle}

\paragraph{Dalton.}

Hi there there should be some text

\paragraph{Rutherford.}

Hi there there should be some text

\paragraph{Bohr.}

Hi there there should be some text

\setcounter{part}{2}
\setcounter{section}{0}
\part*{Bindungslehre}
\addcontentsline{toc}{part}{Bindungslehre}

\section{Kovalente Bindung}

\subsection{Strukturschreibweisen}

\begin{itemize}
	\item Strukturformel
	\item Skelettformel = Lewis-Formel
	\item Gruppenformel
\end{itemize}

\subsection{Struktur und Geometrie von Molekülen, Elektronennegativität, Polarität}

\subsubsection{Orbitale}

\begin{itemize}
	\item s: 2 Elektronen
	\item p: 6 Elektronen
	\item d: 10 Elektronen
	\item f: 14 Elektronen
\end{itemize}

\subsubsection{Elektronenkonfiguration Schreibweise}

C: $1s^2 \ 2s^2 \ 2p^2$ oder $[He] \ 2s^2 \ 2p^2$

\subsection{Zwischenmolekulare Kräfte}

\subsection{Eigenschaften molekularer Stoffe}
\section{Ionenbindung}

\subsection{Struktur und Aufbau von Salzen}

\subsection{Eigenschaften von Salzen}

\paragraph{Schmelzpunkt, Siedepunkt.}

Hoch, bei Raumtemperatur sind alle Salze fest Abhängig von Gitterenergie

\paragraph{Löslichkeit.}

Mehr oder weniger in Wasser löslich, je nach Gitterenergie und Hydratationsenergie. Faustregel: unlöslich, wenn beide Ionen Ladung 2 oder höher haben (Ausnahmen: AgCl, AgI) Nicht löslich in unpolaren Lösungsmitteln

\paragraph{Sonstiges.}

Spröde, in Lösung oder als Schmelze, elektrisch leitfähig

\paragraph{Reaktionen.}

Salzbildung, Elektrolyse, Lösen, Fällen
\section{Metallbindung}

\subsection{Aufbau von Metallen}

\subsection{Eigenschaften von Metallen}

\setcounter{part}{18}
\setcounter{section}{0}
\part*{Reaktionslehre}
\addcontentsline{toc}{part}{Reaktionslehre}

\section{Chemisches Rechnen}

\subsection{Stöchiometrisches Rechnen}

\begin{definition}[mol]
	$1 \ mol = 6.022 \cdot 10^{23}$ Teilchen
\end{definition}

\begin{definition}[m]
	Gewicht absolut in $[g]$
\end{definition}

\begin{definition}[n]
	Teilchenanzahl; in $[mol]$.
\end{definition}

\begin{definition}[M]
	Molare Masse in $[g/mol]$
\end{definition}

\begin{definition}[V]
	Volumen; in	$[l]$
\end{definition}

\begin{definition}[c]
	Konzentration in $[mol/l]$, manchmal $M$ verwendet.
\end{definition}

\begin{definition}[T]
	Konzentration; in $[K]$
\end{definition}

\begin{definition}[p]
	Druck; in $[Pa]$.
\end{definition}


\subsection{Konzentrationsberechnungen}

{\large
	
$$
\begin{array}{rcll}
n                & = & \frac{m}{M}           &\\[0.5em]
M                & = & \frac{m}{n}    &\\[0.5em]
c                 & = & \frac{n}{V} &\\[0.5em]
n                 & = &\frac{V}{V_m}             &\\[0.5em]
1l                & = & 1\ dm^3=0.001\ m^3         &\\[0.5em]
p \cdot v    & = & n \cdot R \cdot T                       &\\[0.5em]
R                & = & 8.31448 \ \frac{J}{K \cdot mol}                       &\\[0.5em]
0\degree C & = & 273.15K             &\\[0.5em]
1\ bar         & = & 10^5 Pa; \ p_{n} = 1\ bar = 1013.25\ hPa & 
\end{array}
$$

}
\section{Kinetik}

\subsection{Grundlagen}

TODO
\section{Chemisches Gleichgewicht}

\subsection{Grundlagen der Thermodynamik}

\begin{definition}[Aktivierungsenergie]
	Energiezufuhr, die eingesetzt werden muss, damit die Reaktion anfängt, abzulaufen.
\end{definition}

\begin{definition}[innere Energie U]
	\leavevmode \\
	$\Delta U = q+w$ oder $\Delta U = \Delta H - p \cdot \Delta V$
	
	$\Delta U$: Änderung der inneren Energie
	
	$q$: als Wärme zugeführte Energie
	
	$w$: als Arbeit zugeführte Energie, kann zum Beispiel Volumenveränderung sein.

\end{definition}

\begin{definition}[Volumenarbeit]
	$w = p \cdot \Delta V$
	
	$w$: Volumenarbeit
	
	$p$: Druck
	
	$\Delta V$: Volumenänderung
	
	$\Delta H$: Enthalpie
	
\end{definition}

\subsection{Enthalpie $\Delta H$}

	Wärmemenge, die während einer Reaktion bei konstanten Druck auf ein System übertragen wird.
	
	\large{
		$\Delta H = \\ \Sigma \Delta H^0_R = \Sigma \Delta H^0_f (Produkte) - \Sigma \Delta H^0_f (Edukte)$
	}

\begin{definition}[Standardbildungsenthalpie]
	Stoffparameter, in den Bindungen gespeicherte Energie, die als Wärme freigesetzt werden kann.
	
	$\Delta H^0_f$ in $kJ/mol$
	
	\begin{note}
		$\Delta H^0_{f,O_2} = 0$
	\end{note}

\end{definition}

\begin{definition}[endotherme Reaktion]
$\Delta H^0_R > 0$
Wärme wird von der Umgebung aufgenommen. Das Reaktionsgemisch kühlt sich ab.
\end{definition}

\begin{definition}[exotherme Reaktion]
$\Delta H^0_R < 0$
Wärme wird an die Umgebung abgegeben. Das Reaktionsgemisch erwärmt sich.
\end{definition}

\subsection{Entropie $\Delta S$}


	Die Entropie ist ein Zustand, die den Ordnungsgrad beschreibt. Hohe Entropie = Hohes Chaos. Alle Reaktionen streben nach einem hohen Entropiegrad. Ludvig Boltzmann.
	
	\large{
		$\Delta S^0_R = \Sigma \Delta S^0 (Produkte) - \Sigma \Delta S^0 (Edukte)$
	}

\begin{definition}[Entropieformel]
	$S = k \cdot ln(W)$
	
	$S$: Entropie
	
	$k=R/N_A=1.38\cdot 10^{-23} J/K$: Boltzmann-Konstante 
	
	$W$: idk man
	
	Die Entropie ist proportional zum Logarithmus der zugänglichen innerer Energie.
\end{definition}

\begin{definition}[Entropieänderung]
	$\Delta S^0_R > 0$: Mehr Entropie
	
	$\Delta S^0_R < 0$: Weniger Entropie
\end{definition}

Generell sind aus entropischer Sicht Reaktionen bevorzugt, welche Unordnung produzieren: \\
$\Delta S_{gesamt} = \Delta S_{System} + \Delta S_{Umgebung}$

Ein spontaner Vorgang ist nur dann möglich, wenn eine Zunahme der Gesamtentropie von System und Umgebung erfolgt.


\subsection{Freie Enthalpie $\Delta G$}

Gesamtenergieumsatz einer chemischen Reaktion.

	\large{
		$\Delta G^0_R = \Sigma \Delta G^0_f (Produkte) - \Sigma \Delta G^0_f (Edukte)$
	}
	
\begin{definition}[endergonische Reaktion]
	Die Reaktion läuft nicht freiwillig ab.
	
	$\Delta G^0_R > 0$
\end{definition}

\begin{definition}[exergonische Reaktion]
	Die Reaktion läuft freiwillig ab, evtl. Aktivierungsenergie nötig.
	
	$\Delta G^0_R < 0$
\end{definition}

\subsection{Die Gibbs-Energie}

\large{
	$\Delta G = \Delta H - T\cdot \Delta S$
}
	
	$\Delta G$: Freie Enthalpie
	
	$\Delta H$: Enthalpie
	
	$T$: Temperatur
	
	$\Delta S$: Entropie \\

	Exotherme Reaktionen, die Unordnung produzieren:   H\textless0 S\textgreater0    \\
	
	Endotherme Reaktionen, die Unordnung produzieren  H\textgreater0 S\textgreater0 \\
	
	Exotherme Reaktionen, die Ordnung produzieren     H\textless0 S\textless0      

\subsection{Massenwirkungsgesetz}

%TODO: Auf ChiCD auf Abschnitt 6.4







\section{Ozon}
\section{Säure-Base Reaktionen}

\subsection{Definition nach Brönsted}

\textit{Säuren sind Protonenspender, Basen sind Protonenakzeptoren.}

\subsection{Säure-Base Reaktionen (Protolyse)}

\begin{itemize}
	\item Nichtmetalloxide erzeugen saure Lösungen, Metalloxide erzeugen basische Lösungen.
	\item Saure und basische Lösungen leiten Strom: Die Lösungen enthalten Ionen.
	\item Säuren enthalten Wasserstoffatome, Basen enthalten stark elektronegative Atome mit nichtbindenden Elektronenpaaren.
\end{itemize}

\begin{definition}[Protolyse]
	Säure-Base-Reaktion: Ein Stoff gibt ein Proton ab, ein Stoff nimmt ein Proton auf.
\end{definition}

\begin{definition}[Beispiel Säure]
	{\large
	$HCl$
}
\end{definition}

\begin{definition}[Beispiel Säurereaktion]
	{\large
		$HCl + H_2O \rightarrow Cl^\ominus + H_3O^\oplus$
	}
\end{definition}

\begin{definition}[Beispiel Base]
	{\large
		$NH_3$
	}
\end{definition}

\begin{definition}[Beispiel Basereaktion]
	{\large
		 $NH_3 + H_2O \rightarrow NH_4^\oplus + OH^\ominus$
	}
\end{definition}

\begin{definition}[Hydronium-Ion]
	{\large
		$H_3O^\oplus$
	}
\end{definition}

\begin{definition}[Hydroxid-Ion]
	{\large
		$OH^\ominus$
	}
\end{definition}

\begin{definition}[Ampholyte]
	Stoffe, die die Funktion einer Säure und einer Base einnehmen können. Zum Beispiel {\large $H_2O$}
\end{definition}

\begin{definition}[Autoprotolyse von Wasser]\leavevmode \\

{\large
	$H_2O + H_2O \rightarrow H_3O^\oplus + OH^\ominus$	
}

\end{definition}

\subsection{pH Berechnungen}

\subsubsection{Allgemein}

Das Massenwirkungsgesetz (\ref{mwg}) auf Autoprotolyse anwenden:

{\large
	\begin{equation}
	K_c=\frac{c^c(C) \cdot c^d(D)}{c^a(A) \cdot c^b(B)}=\frac{c(H_3O^\oplus) \cdot c(OH^\ominus)}{c^2(H_2O)}
	\end{equation}
}

Man kann nun annehmen, dass $c(H_2O)$ konstant ist im Reaktionssystem, wenn nur kleine Mengen an Säuren und Basen reagieren $\rightarrow$ \textit{Ionenprodukt}.

\begin{definition}[Ionenprodukt]\leavevmode \\
	
	{\large
		\begin{equation}
		\begin{split}
		K_w & = K_c \cdot c^2(H_2O) \\
		       & = c(H_3O^\oplus) \cdot c(OH^\ominus) \\
		       & = K_S \cdot K_W \\
		       & = 10^{-14} \ M^2
		\end{split}
		\end{equation}
		
		$K_w=K_c \cdot c^2(H_2O)=c(H_3O^\oplus) \cdot c(OH^\ominus)=10^{-14}\ M^2$
	}
	(Bei $25\degree C$)
	
\end{definition}

\begin{definition}[Saure Lösung]
	{\large
		$c(H3O^\oplus) > 10^{-7} M$
	}
\end{definition}

\begin{definition}[Neutrale Lösung]
	{\large
		$c(H3O^\oplus) = 10^{-7} M$
	}
\end{definition}

\begin{definition}[Basische Lösung]
	{\large
		$c(H3O^\oplus) < 10^{-7} M$
	}
\end{definition}

\begin{definition}[pH-Wert]
	Negativer 10-er Logarithmus der Konzentration der Hydroniumionen.
	
	{\large
		\begin{equation}
			\begin{split}
				pH		&=-log_{10}\ c(H_3O^\oplus) \\
						  &= 14-pOH
			\end{split}
		\end{equation}
	}
\end{definition}

\begin{definition}[pOH-Wert]
	Negativer 10-er Logarithmus der Konzentration der Hydroxidionen.
	
	{\large
		\begin{equation}
			\begin{split}
				pOH&=-log_{10}\ c(OH^\ominus) \\
				&=14-pH
			\end{split}
		\end{equation}
	}
\end{definition}

Skala reicht von 0 bis 14. 0 ist sauer, 7 ist neutral, 14 ist basisch.

\begin{definition}[Säurekonstante]
	Bei einer vorhandenen Protolyse, bei der eine Säure mit Wasser reagiert, wird die Gleichgewichtskonstante ($K_c$) mit der Konzentration des Wassers ($c(H_2O)$) multipliziert, da sich diese ja konstant verhält.
	
	{\large
		\begin{equation}
		\begin{split}
		K_S & =K_C \cdot c(H_2O) = \frac{c(H_3O^\oplus) \cdot c(X^\ominus)}{c(HX)} \\
		pK_S & = -log_{10}\ K_S
		\end{split}
		\end{equation}
	}
	
	Hoher $K_S$-Wert oder niedriger $pK_S$-Wert: Vollständige Protonenabgabe.
	
	Tiefer $K_S$-Wert oder hoher $pK_S$-Wert: Unvollständige Protonenabgabe.
\end{definition}

\begin{definition}[Basenkonstante]
	Das gleiche wie Säurekonstante, einfach für basische Reaktionen.
	
	{\large
		\begin{equation}
		\begin{split}
		K_B & =K_C \cdot c(H_2O) = \frac{c(OH^\ominus) \cdot c(HB^\oplus)}{c(B)} \\
		pK_B & = -log_{10}\ K_B
		\end{split}
		\end{equation}
	}
	
	Hoher $K_B$-Wert oder niedriger $pK_B$-Wert: Vollständige Protonenaufnahme.
	
	Tiefer $K_B$-Wert oder hoher $pK_B$-Wert: Unvollständige Protonenaufnahme.
\end{definition}

{\large
\begin{equation}
	pK_S + pK_B = 14.
\end{equation}
}

\begin{definition}[Starke Säure]
	Sie geben ihre Protonen vollständig ab. \\
	{\large $pK_S < 0$}
\end{definition}

\begin{definition}[Starke Base]
	Sie nehmen  Protonen vollständig auf. \\
	{\large $pK_B < 0$}
\end{definition}

\subsubsection{pH-Berechnung mit starken Säuren/Basen}

Für $HCl\ (pK_S=-6; \ c(HCl) = 10^{-2} M)$:

$HCl + H_2O \rightarrow H_3O^\oplus + Cl^\ominus$

$c_{GGW}(H_3O^\oplus) = c_0(HCl); \ K_S = 10^{-2}; \ pK_S = 2$

{\large
	\begin{equation}
	\label{säure}
		\begin{split}
			 K_S=c_{GGW}(H_3O^\oplus) &= c_0(HX)
		\end{split}
	\end{equation}	
}

\ref{säure}: Starke Säure

{\large
	\begin{equation}
	\label{base}
	\begin{split}
	K_B=c_{GGW}(OH^\ominus) &= c_0(B)
	\end{split}
	\end{equation}	
}

\ref{base}: Starke Säure

\subsubsection{pH-Berechnung mit schwachen Säuren/Basen}


\begin{alignat*}{5}
	&HX \ \ \ +\ \ \ &&H_2O\ \ \rightarrow \ \ &&H_3O^\oplus \ \ \ +\ \ \ &&H^\ominus && \\
	&c_0 - x && konst. && x && x &&
\end{alignat*}

\begin{alignat*}{5}
	&B \ \ \ +\ \ \ &&H_2O\ \ \rightarrow \ \ &&OH^\ominus \ \ \ +\ \ \ &&HB && \\
	&c_0 - x && konst. && x && x &&
\end{alignat*}

{\large
	\begin{equation}
		K_{\{S|B\}} = \frac{x^2}{c_0-x}
	\end{equation}	
}

{\large
	\begin{equation}
	x^2 + x\cdot K_{\{S|B\}} - c_0 \cdot K_{\{S|B\}} = 0
	\end{equation}
}

{\large
	\begin{equation}
	\begin{split}
		\{pH\ |\ pOH\}&=-log_{10}\ x \\
		&=14-\{pOH\  |\  pH\} \\
	\end{split}
	\end{equation}
}

\subsection{Neutralisationen}

\subsection{Titrationen}

\subsection{Puffer}

%http://paedubucher.ch/passerelle/chemie/heft_107.pdf
\section{Redox-Reaktionen}

\subsection{Grundlagen}

\begin{definition}[Redoxreaktion]
	Elektronenaustauschreaktion. Bei einer Redoxreaktion werden die Oxidationszahlen verändert.
\end{definition}

\begin{definition}[Oxidationszahl]
	Beschreibt formell die Ladung eines Atoms.
	
	\begin{itemize}
		\item Atome bei Elementarbindungen besitzen die Ladung 0
		\item Bei kovalenten Bindungen werden die Elektronen dem elektronegativeren Partner zugeordnet
		\item Summe der Oxidationszahlen muss der Gesamtladung der Bindung entsprechen
		\item $H$ hat immer die Oxidationszahl $+I$ ($H^{+I}$) ausser bei $H_2\ (=2\ H^0)$
		\item $O$ hat immer die Oxidationszahl $-II$ ($O^{-II}$) ausser bei $O_2\ (=2\ O^0)$, Peroxiden (\chemfig{O-[,.7]O}) und Fluorverbindungen.
	\end{itemize}
\end{definition}

\begin{definition}[Reduktion]
	Elektronenaufnahme; OZ wird verringert. Beispiel:\\
	$4\ O^0 + 8\ e^\ominus\rightarrow 4\ O{-II}$
\end{definition}

\begin{definition}[Oxidation]
	Elektronenabgabe; OZ wird vergrössert. Beispiel:\\
	$C^{-IV} \rightarrow C^{+IV} + 8\ e^\ominus$
\end{definition}

\begin{definition}[Oxidationsmittel]
	Stoffe, die reduzieren, d. h. Elektronen aufnehmen.
\end{definition}

\begin{definition}[Reduktionsmittel]
	Stoffe, die oxidieren, d. h. Elektronen abgeben.
\end{definition}

\subsection{Redoxreihe}

Ermittlung von stärkeren und schwächeren Reaktionspartnern.\\

Spontane Reaktion: Links oben - Rechts unten.

\subsection{Galvanisches Element}

Funktion: Gewinnung von Energie aus Redoxreaktionen, Umwandlung von chemischer Energie in elektrische Energie.

\begin{itemize}
	\item Zwei Metalle: \textit{Edleres Metall} und \textit{weniger edles Metall}.
	\item Edleres Metall wird reduziert, unedles Metall wird oxidiert.
\end{itemize}

\begin{definition}[Anode]
	Pol, an dem die Oxidation stattfindet (Elektronenabgabe); negativer Pol; unedles Metall; Reduktionsmittel; z. B. $Zn$; $Zn \rightarrow Zn^{2\oplus}\ +\ 2e^\ominus$
\end{definition}

\begin{definition}[Kathode]
	Pol, an dem die Reduktion stattfindet (Elektronenaufnahme); positiver Pol; edles Metall; Oxidationsmittel; z. B. $Cu^{2\oplus}$; $Cu^{2\oplus}\ + \ 2e^\ominus \rightarrow Cu$
\end{definition}

\begin{definition}[Stromfluss]
	Anode $\rightarrow$ Kathode
\end{definition}

\begin{definition}[Membran]
	Halbdurchlässige Membran zwischen Anode und Kathode, sodass Ionen passieren können.
\end{definition}

\begin{definition}[Standardreduktionspotential]
	Gibt die Edelheit eines Stoffes an. Platinelektrode, die von $25\degree C$ heissen $1M$-Salzsäurelösung umströmt wird weist eine Spannung von $0$ auf.
	Edelmetalle: $>0V$ Nicht von Antoniadis behandelt (TODO prüfen).
\end{definition}


Nasszelle gehabt bei Antoniadis? TODO.

\subsection{Batterien}

% ein galv element

\subsubsection{$Cu-Zn$}

Anode: $Zn \rightarrow Zn^{2\oplus} + 2e^\ominus$\\

Kathode: $Cu^{2\oplus} + 2e^\ominus \rightarrow Cu$

\subsubsection{$Zn-C$}

Anode: $Zn \rightarrow Zn^{2\oplus} + 2e^\ominus$

$Zn^{2\oplus}+2NH_3+2Cl^\ominus \rightarrow [Zn(NH_3)_2]^{2\oplus}+2Cl^\ominus$\\

Kathode: $2NH_4^\oplus + 2e^\ominus + 2Cl^\ominus \rightarrow 2NH_3 + H_2 + 2Cl^\ominus$

$H_2 + 2MnO_2 \rightarrow MnO_3 + H_2O$

\subsubsection{Alkali-$Mn$}

Anode: $Zn + 2OH^\ominus \rightarrow Zn(OH)_2 + 2e^\ominus$\\

Kathode: $2MnO_2 + H_2O + 2e^\ominus \rightarrow Mn_2O_3+2OH^\ominus$

\subsubsection{$Zn$-Luft}

Anode: $2 Zn \rightarrow 2 Zn^\oplus + 4e^\ominus$\\

Kathode: $O_2 + H_2O + 4e^\ominus \rightarrow 4OH^\ominus$


\subsubsection{$Li-MnO_2$}

Anode: $Li \rightarrow Li^\oplus + e^\ominus$\\

Kathode: $Mn(IV)O_2 + Li^\oplus + e^\ominus \rightarrow Li^\oplus + Mn(III)O_2$\\

Nicht aufladbar, weil: $Li^\oplus + Mn(III)O_2 \rightarrow Mn(IV)O_2 + Li^\oplus + e^\ominus$ nicht möglich.

\subsection{Akkus}

% ein galv element

\subsubsection{Blei-Akku}

\subsubsection{Lithium-Ionen-Akku}


\subsection{Brennstoffzelle}

% ein galv element

\subsubsection{PEM (Proton Exchange Membrane)}

\subsubsection{Alkalische Brennstoffzelle}

\subsubsection{Direkt-Methanol Brennstoffzelle}


\subsection{Elektrolyse}

Erzwungene Redoxreaktion. Die Anode ist neuerdings der Pluspol, die Kathode der Minuspol. Anlegen einer Spannung bei der Kathode. Umwandlung elektrischer Energie in chemische Energie.

{\large
	\begin{equation}
		\label{elektrolyse}
		\begin{split}
			Gesamt:\ &2\ H_2O \rightarrow 2\ H_2 + O_2 \\
			Reduktion:\ &2\ H^{+I} \rightarrow 2\ e^\ominus + 2\ H^0 \\
			Oxidation:\ &O^{-II} + 2\ e^\ominus \rightarrow 2\ O^0
		\end{split}
	\end{equation}
}

\ref{elektrolyse}: Beispiel: Elektrolyse von Wasser. \\

\begin{definition}[Galvanisieren]
	TODO ergänzen.
\end{definition}


\setcounter{part}{15}
\setcounter{section}{0}
\part*{Organische Chemie}
\addcontentsline{toc}{part}{Organische Chemie}

\section{Kohlenwasserstoffe}

\subsection{Alkane, Alkene, Alkine, Aromaten}

\subsubsection{Allgemein}

\begin{definition}[Alkane]
	Weisen ausschliesslich Einfachbindungen auf.
\end{definition}

\begin{definition}[Alkene]
	Weisen mindestens eine Doppelbindung auf.
\end{definition}

\begin{definition}[Alkine]
	Weisen mindestens eine Dreifachbindung auf.
\end{definition}

\begin{definition}[Aromate]
	Ringförmige Moleküle, welche über eine ungerade Anzahl alternierender Doppelbindungen verfügen
\end{definition}

\subsubsection{Wichtige Vertreter}

\begin{definition}[Vertreter Alkane]
	Wasser, Ethanol
\end{definition}

\begin{definition}[Vertreter Alkene]
	Werden verwendet für Synthetisierung von Alkohol, Glycerin.
\end{definition}

\begin{definition}[Vertreter Alkine]
	Fungizide
\end{definition}

\begin{definition}[Vertreter Aromate]
	Aminosäuren wie z. B. Phenylalnin. \\
	
	\setatomsep{2.2em}
	\chemfig[][scale=0.8]{HO-[:30](=[:90]O)-[:-30](-[:-90]NH2)-[:30]-[:-30]*6(-=-=-=)}
\end{definition}

\subsubsection{Aromate}

\begin{definition}[Aromat]
	(Zyklische) Kohlenwasserstoffe, die ein delokalisiertes Elektronensystem besitzen.
\end{definition}

\begin{definition}[Hückel-Regel]
	Aromatische Moleküle mit 4n + 2 Elektronen enthalten ein delokalisiertes Elektronensystem.
\end{definition}

\begin{definition}[Benzol]
	$C_6H_6$, Ring von 6 $C$-Atomen mit einem delokalisiertem Elektronensystem. \\
	
	\chemfig{**6(------)}
\end{definition}

\begin{definition}[Delokalisiertes Elektronensystem]
	Elektronen verteilen sich gleichmässig über alle Kohlenstoffatome hinweg.
\end{definition}

\begin{definition}[Kekulé-Formel]
	In Wirklichkeit nicht so. Delokalisiertes Elektronensystem ist aktuell; $\Delta H^0_R$ wäre viel zu krass (viel zu klein, ist Wirklichkeit etwas grösser, d. H. weniger Energie freigesetzt).
	\leavevmode \\
	
	\chemfig{*6(-=-=-=)}
\end{definition}

\subsection{Isomere}

\begin{definition}[Chiralität]
	Ein $C$-Atom hat vier verschiedene Bindungspartner. Chirale Atome in einem Molekül sind Voraussetzung für Enantiomere (1 chirales Atom) und Diastereomere (mindestens 2 chirale Atome).
\end{definition}

\begin{itemize}
	\item \textbf{Konstitutionsisomere}: Moleküle haben gleiche Summenformel, aber andere Verknüpfung von Atomen, andere Namen.
	\item \textbf{Stereoisomere}: Moleküle haben gleiche Summenformel, gleiche Verknüpfung und gleichen Namen, unterschiedliche räumliche Anordnung.
		\subitem \textbf{Konfigurationsisomere}: 
		\subitem $\rightarrow$ \textbf{Enantiomere}: Verhalten sich wie Spiegelbilder, weisen aber keine Symmetrieebene auf. 1 chirales Atom.
		\subitem $\rightarrow$ \textbf{Diastereomere}: Mindestens 2 chirale Atome.
		\subitem $\rightarrow$ \textbf{cis-trans-Isomere}: an einer $C=C$-Doppelbindung sind je zwei verschiedene Substituenten entgegengesetzt (trans) oder auf einer Seite (cis) gebunden.
		\subitem \textbf{Konformationsisomere}: Können durch Rotation ineinander überführt werden.
\end{itemize}

\subsection{Nomenklatur}

Beispiel: \textit{2-Methylpropanol}

\subsubsection{Vorgehen}

\begin{itemize}
	\item \textit{Stammname}: Längste C-Kette ergibt den Stammnamen. So nummerieren, dass die längste Kette gewählt wird und die Substituenten eine möglichst kleine Zahl besitzen
	\item \textit{Funktionelle Gruppen}: Funktionelle Gruppen bestimmen
	\item \textit{Substituenten}: Substituenten bestimmen
	\item \textit{Doppel- und Dreifachbindungen}: Doppel- und Dreifachbindungen mit -en und -in bezeichnen.
\end{itemize}

Format: Substituenten, Stammname, Zwei- / Dreifachbindungen, Funktionelle Gruppe.
\subsubsection{Stammnamen}

\begin{center}
	\begin{tabular}{ l  l  l }
		\textbf{Zahl} & \textbf{Stammname} & \textbf{Zahlwort} \\
		1 & Meth & \\
		2 & Eth & Di\\
		3 & Prop & Tri\\
		4 & But & Tetra\\
		5 & Pent & Penta\\
		6 & Hex & Hexa\\
		7 & Hept & Hepta\\
		8 & Oct & Octa\\
		9 & Non &Nona\\
		10 & Dec &Deca\\
	\end{tabular}
\end{center}

\subsubsection{Substituenten}

Bei Einfachbindung: \{Name\}-yl anhängen.

Bei Doppelbindung:  \{Name\}-ylen anhängen.

Bei Dreifachbindung:  \{Name\}-ylidin anhängen.

Bei Bei Cycloverbindungen:  Cyclo-\{Name\} einfügen.

\begin{center}
	\begin{tabular}{ l  l  l }
		\textbf{Substituent} & \textbf{Name} \\
		$-CH_3$ & Methyl \\
		$-C_2H_5$ & Ethyl \\
		$-C_nH_{2n+1}$ & \{n-Stamm\}yl \\
		$-F$ & Fluor \\
		$-Cl$ & Chlor \\
		$-Br$ & Brom \\
		$-I$ & Iod \\
		$-NH_2$ & Amino \\
		$-OH$ & Hydroxy \\
		$-NO_2$ & Nitro \\
		$-SH$ & Mercapto \\
	\end{tabular}
\end{center}

\begin{note}
	Substituenten, Funktionelle Gruppen und Doppel-/Dreifachbindungen werden immer mit dem Format \\
	
	$\{a_0,...,a_n\}-\{Zahlwort\}\{Name\}$, z. B. 3,4-dien \\
	
	bezeichnet. 
	$a$: Der Ort auf der C-Kette, wo gebunden.
\end{note}

\subsubsection{Funktionelle Gruppen}

\begin{definition}[Halogene Wasserstoffe]
	Enthalten mindestens ein Halogenatom $F,\ Cl,\ Br,\ I$. Struktur: $R-X$
	
	Als Name für die funktionelle Gruppe dient der Name des Halogenatoms. 
\end{definition}

\begin{definition}[Reihenfolge der Verbindungen]
	\textit{CCCAKATAE} \\
	
	\setatomsep{2.2em}
	\begin{tabularx}{.5\textwidth}{X l}
		\textbf{Name + Benennung} & \textbf{Formel} \vspace{2em}\\
		
		\vspace{-2em} 
		Carbonsäure
		
		 \textit{\{Stamm\}säure}& 
		\chemfig[][scale=.8]{C(=[:90]\lewis{13,O})(-[:-150]R)-[:-30]OH} \vspace{2em} \\
		
		\vspace{-2em} 
		Carbonsäureester
		
		\textit{\{Stamm\}säure-\{R'\}ester} & 
		\chemfig[][scale=.8]{C(=[:90]\lewis{13,O})(-[:-150]R)-[:-30]\lewis{57,O}-[:30]R'} \vspace{2em} \\
		
		\vspace{-2em} 
		Carbonsäureamide
		
		\textit{\{Stamm\}säure-N-\{R'\}yl-N-\{R''\}ylamid}
		
		\textit{\{Stamm\}säure-N-N-di\{R'\}ylamid} 
		
		Kann sein, dass $R'=R''=H$ ist, dann ylamid-Teil weglassen.& 
		\chemfig[][scale=.8]{C(=[:90]\lewis{13,O})(-[:-150]R)-[:-30]N(-[:30]R')-[:-90]R''} \vspace{2em} \\
	
		\vspace{-2em} 
		Aldehyde
		
		\textit{\{Stamm\}al}& 
		\chemfig[][scale=.8]{C(=[:90]\lewis{13,O})(-[:-150]R)-[:-30]H} \vspace{2em} \\
		
		\vspace{-2em} 
		Ketone
		
		\textit{\{Stamm\}on} &
		\chemfig[][scale=.8]{C(=[:90]\lewis{13,O})(-[:-150]R)-[:-30]R'} \vspace{2em} \\
		
		\vspace{-2em} 
		Alkohole
		
		\textit{\{Stamm\}ol} &
		\chemfig[][scale=.8]{R-OH} \vspace{2em} \\
		
		\vspace{-2em} 
		Thiole
		
		\textit{\{Stamm\}thiol} &
		\chemfig[][scale=.8]{R-SH} \vspace{2em} \\
		
		\vspace{-2em} 
		Amine
		
		\textit{\{Stamm\}ylamin} &
		\chemfig[][scale=.8]{R-NH_2} \vspace{2em} \\
		
		\vspace{-2em} 
		Ether
		
		\textit{\{R'\}yl-\{R''\}ylether} &
		\chemfig[][scale=.8]{R-[:30]\lewis{13,O}-[:-30]R} \vspace{2em} \\
	\end{tabularx}
\end{definition}

\subsubsection{Cycloalkane}

Stamm: \textit{Cyclo\{Stamm\}\{en/in\} }

Substituent: \textit{Cyclo\{Gruppe\}yl-\{Stamm\}\{en/in\} } \\

\setatomsep{2.2em}
\chemfig{*6(-=-=-=)} (Benzol) \ \
\ \chemfig{*6(------)} (Cyclohexan)

\subsubsection{Aromate}

Als Stamm: \{Substituenten\}benzol

Als Substituent: Phenyl-\{Stamm\} \\

\setatomsep{2.2em}
\chemfig{**6(----(-[:90]C(=[:150]\lewis{24,O})-[:30]OH)--)} %benzoesäure 
(Phenylmethansäure)
\chemfig{**6(----(-[:90]OH)--)} %phenol
(\textbf{Phenol}) 
\chemfig{**6(----(-[:90]NH_2)--)} %phenol
(Phenylamin oder Anilin) 








\section{Erdöl}

\subsection{Allgemein}

\begin{definition}[Erdöl]
	C-Verbindungen, aus Rohöl. Formel:
	
{\large
	\begin{equation}
		C_{n}H_{2n+2}		
	\end{equation}
}

\begin{center}
	\begin{tabular}{ l  l }
		\textbf{Länge} & \textbf{Name} \\
		$1-4$ & Raffineriegas \\
		$5-7$ & Leichtbenzin \\
		$6-10$ & Schwerbenzin \\
		$10-16$ & Kerosin \\
		$12-18$ & Paraffinöl \\
		$14-20$ & Schweröl \\
		$>20$ & Destillationsrückstand
	\end{tabular}
\end{center}
\end{definition}

\subsection{Fördern von Erdöl}

Erdöl entsteht beim Zusammenpressen von abgesunkenen Plankton. Suchen durch seismische Messungen (Reflektionsseismik) und Computermodellen. Förderung durch Abpumpen oder Fracking (Einpumpen von Lösestoffen).

\subsection{Probleme von Erdöl}

Ertrag entspricht nicht der Nachfrage; Weiterverarbeitung des Erdöls.

\subsection{Cracken}

\begin{definition}[Sinn]
	Lange C-Ketten kürzer machen.
\end{definition}

\begin{definition}[Vorgang]
	Erhitzen + Katalysator (meistens Eisenwatte)
\end{definition}

\begin{definition}[Bromwasser Indikator]
	Verfärbt sich bei Kontakt von reaktionsfreudigen Partnern wie zum Beispiel Alkene, welche man in Kondensat findet und nicht in langkettigen Ölen.
\end{definition}

\begin{center}
	\begin{tabular}{ l  l }
		\textbf{Produkt} & \textbf{Edukt} \\
		Alkane & kleinere Alkane, Cycloalkane \\
		Cycloalkane & Alkane, verzweigte Alkene \\
		Alkene & kleinere Aklene \\
		Alkine & verzweigte Alkane und Alkine
	\end{tabular}
\end{center}

\subsubsection{Coken}

Lange, $> 500^{\degree}$

\subsubsection{Katalytisches Cracken}

Am weitesten verbreitet, scharf katalytisch und $480^{\degree} C -600^{\degree} C$

\subsubsection{Hydrocracken}

Mild katalytisch und $300^{\degree} C -450^{\degree} C$

\subsection{Entschweflung}

\begin{definition}[Ziel]
	Organische Schwefelverbindungen $H_2S$ und $SO_2$ entfernen, sodass Katalysator besser funktioniert.
\end{definition}

\begin{definition}[Katalytische Hydrierung]
	Schädlicher Schwefelwasserstoff $H_2S$ wird zur weiterverarbeitung in Schwefeloxid $SO_2$ umgesetzt.
	
	{\large\begin{equation}
			H_2S + 2\ O_2 \rightarrow SO_2 + 2\ H_2O
		\end{equation}}
\end{definition}

\begin{definition}[Claus-Prozess]
	Schwefelwasserstoff + Schwefeloxid entfernen und zu elementaren Schwefelstoff umwandeln, sodass Katalysator besser funktioniert.
	
	{\large\begin{equation}
		8\ SO_2 + 16\ H_2S \rightarrow 3\ S_8 + 16\ H_2O
		\end{equation}}
\end{definition}

\subsection{Saurer Regen}

Entstehung: Wenn $SO_2$ in die Luft gelangt. Schwefelsäure ist eine Säure, die Sachen zerfrisst.

\begin{definition}[Verbrennung]\leavevmode\\
	{\large\begin{equation}
		H_2S + 2\ O_2 \rightarrow SO_2 + 2\ H_2O
		\end{equation}}
\end{definition}

\begin{definition}[Entstehung Schwefeltrioxid]\leavevmode\\
	{\large\begin{equation}
		2\ SO_2 + O_2 \rightarrow 2\ SO_3 + 2\ H_2O
		\end{equation}}
\end{definition}

\begin{definition}[Entstehung Schwefelsäure]\leavevmode\\
	{\large\begin{equation}
		SO_3 + H_2O \rightarrow H_2SO_4
		\end{equation}}
\end{definition}

\begin{definition}[Reaktion Schwefelsäure]\leavevmode\\
	{\large\begin{equation}
		H_2SO_4 + H_2O \rightarrow HSO_4^\ominus + H_3O^\oplus
		\end{equation}}
\end{definition}

\begin{definition}[Schweflige Säure]\leavevmode\\
	{\large\begin{equation}
		H_2SO_3
		\end{equation}}
\end{definition}

\subsection{Oktanzahl \& Cetanzahl}

\begin{definition}[Oktanzahl]
	Mass für den Widerstand gegen Selbstentzündung, Klopffestigkeit, bei Benzin. \\
	
	Tiefe OZ: Hohe Zündwilligkeit
	
	Hohe OZ: Geringe Zündwilligkeit\\
	
	$OZ=0$: n-Heptan; Hohe Zündwilligkeit
	
	$OZ=100$: Isooctan, 2,2,4-Trimethylpentan; Tiefe Zündwilligkeit\\
	
	$95\% = 5\%\  n-Heptan + 95\%\ Isooctan$
\end{definition}

\begin{definition}[Cetanzahl]
	Mass für den Widerstand gegen Selbstentzündung, Klopffestigkeit, bei Diesel. \\
	
	Tiefe OZ: Geringe Zündwilligkeit
	
	Hohe OZ: Hohe Zündwilligkeit\\
	
	$CZ=0$: Methylnaphtalin; Tiefe Zündwilligkeit
	
	$CZ=100$: Cetan, n-Hexadecan; Hohe Zündwilligkeit\\
	
	$51\% = 49\%\  Methylnaphtalin + 95\%\ Cetan$
\end{definition}

TODO vlt Formeln?

\subsection{Raffination}

Umwandlung von Verbindungen um einen höheren Verzweigungsgrad und Ungesättigtheitsgrad zu erlangen.

\begin{definition}[Isomerisieren]
	Reaktionen, die zu Isomeren der ursprünglichen Moleküle führen.
\end{definition}

\begin{definition}[Reformieren]
	Cyclisieren und Abspalten und Wasserstoffatomen (to reform = neu bilden).
\end{definition}

\section{Kunststoffe}

\subsection{Polymere}

\subsection{Einteilung}

% arten polyamide polyester auch beschreiben kurz in 2. einteilung

\subsection{Polymerisation}

% ist synthese

\subsection{Polykondensation}

% ist synthese

\subsubsection{Carbonsäureamidbildung}

%carbonsäureamidbildung = entstehung polyamide

\subsubsection{Esterbildung}

%Besterbildung = entstehung polyester

\subsection{Polyaddition}

% ist synthese

\subsection{Spezielle Materialien}

\subsubsection{PLA (Polylactid)}
%% PLA (seiet 8)

\subsubsection{PVA (Polyvinylalkohol)}
%% PVA (löslich seite 13)

\subsubsection{Superabsorber}

\section{Aminosäuren und Proteine}


{\Large Mit <3 gemacht von Max Mathys}

\vspace{.7em}

\textbf{Schule}: MNG Rämibühl
\vspace{.2em}

\textbf{Jahr}: 2016
\vspace{.2em}

\textbf{Lehrerin}: Antoniadis

\label{lastpage}

\end{document}


% ------------------------------------------------------------------------------------------------ %