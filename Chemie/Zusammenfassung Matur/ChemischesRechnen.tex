\section{Chemisches Rechnen}

\subsection{Stöchiometrisches Rechnen}

\begin{definition}[mol]
	$1 \ mol = 6.022 \cdot 10^{23}$ Teilchen
\end{definition}

\begin{definition}[n]
	Teilchenanzahl; in $mol$.
\end{definition}

\begin{definition}[m]
	Gewicht absolut; in g.
\end{definition}

\begin{definition}[M]
	Gewicht pro mol; in $g/mol$
\end{definition}

\begin{definition}[V]

	Volumen; in	$l$; $1l=1\ dm^3=0.001\ m^3$
\end{definition}

\begin{definition}[c]
	Konzentration; in $mol/l$
\end{definition}

\begin{definition}[T]
	Konzentration; in $K$
	
	$0\degree C = 273.15K$
\end{definition}

\subsection{Konzentrationsberechnungen}

\large{
	$n=\frac{m}{M}$
	\\ \\
	$c=\frac{n}{V}$
	\\ \\
	$n = \frac{V}{V_m}$
	\\ \\
	$ p \cdot v = n \cdot R \cdot T $
	\\ \\
	$R = 8.31448 \ \frac{J}{K \cdot mol}$
}