\section{Redox-Reaktionen}

\subsection{Grundlagen}

\begin{definition}[Redoxreaktion]
	Elektronenaustauschreaktion. Bei einer Redoxreaktion werden die Oxidationszahlen verändert.
\end{definition}

\begin{definition}[Oxidationszahl]
	Beschreibt formell die Ladung eines Atoms.
	
	\begin{itemize}
		\item Atome bei Elementarbindungen besitzen die Ladung 0
		\item Bei kovalenten Bindungen werden die Elektronen dem elektronegativeren Partner zugeordnet
		\item Summe der Oxidationszahlen muss der Gesamtladung der Bindung entsprechen
		\item $H$ hat immer die Oxidationszahl $+I$ ($H^{+I}$) ausser bei $H_2\ (=2\ H^0)$
		\item $O$ hat immer die Oxidationszahl $-II$ ($O^{-II}$) ausser bei $O_2\ (=2\ O^0)$, Peroxiden (\chemfig{O-[,.7]O}) und Fluorverbindungen.
	\end{itemize}
\end{definition}

\begin{definition}[Reduktion]
	Elektronenaufnahme; OZ wird verringert. Beispiel:\\
	$4\ O^0 + 8\ e^\ominus\rightarrow 4\ O{-II}$
\end{definition}

\begin{definition}[Oxidation]
	Elektronenabgabe; OZ wird vergrössert. Beispiel:\\
	$C^{-IV} \rightarrow C^{+IV} + 8\ e^\ominus$
\end{definition}

\begin{definition}[Oxidationsmittel]
	Stoffe, die reduzieren, d. h. Elektronen aufnehmen.
\end{definition}

\begin{definition}[Reduktionsmittel]
	Stoffe, die oxidieren, d. h. Elektronen abgeben.
\end{definition}

\subsection{Redoxreihe}

Ermittlung von stärkeren und schwächeren Reaktionspartnern.\\

Spontane Reaktion: Links oben - Rechts unten.

\subsection{Galvanisches Element}

Funktion: Gewinnung von Energie aus Redoxreaktionen, Umwandlung von chemischer Energie in elektrische Energie.

\begin{itemize}
	\item Zwei Metalle: \textit{Edleres Metall} und \textit{weniger edles Metall}.
	\item Edleres Metall wird reduziert, unedles Metall wird oxidiert.
\end{itemize}

\begin{definition}[Anode]
	Pol, an dem die Oxidation stattfindet (Elektronenabgabe); negativer Pol; unedles Metall; Reduktionsmittel; z. B. $Zn$; $Zn \rightarrow Zn^{2\oplus}\ +\ 2e^\ominus$
\end{definition}

\begin{definition}[Kathode]
	Pol, an dem die Reduktion stattfindet (Elektronenaufnahme); positiver Pol; edles Metall; Oxidationsmittel; z. B. $Cu^{2\oplus}$; $Cu^{2\oplus}\ + \ 2e^\ominus \rightarrow Cu$
\end{definition}

\begin{definition}[Stromfluss]
	Anode $\rightarrow$ Kathode
\end{definition}

\begin{definition}[Membran]
	Halbdurchlässige Membran zwischen Anode und Kathode, sodass Ionen passieren können.
\end{definition}

\begin{definition}[Standardreduktionspotential]
	Gibt die Edelheit eines Stoffes an. Platinelektrode, die von $25\degree C$ heissen $1M$-Salzsäurelösung umströmt wird weist eine Spannung von $0$ auf.
	Edelmetalle: $>0V$ Nicht von Antoniadis behandelt (TODO prüfen).
\end{definition}


Nasszelle gehabt bei Antoniadis? TODO.

\subsection{Batterien}

\begin{itemize}
	\item Bei der Anode wird das erste Element oxidiert; die Elektronen werden weggenommen, ausser bei Alkali-$Mn$ (da dieses ein Elektrolyt besitzt).
	\item Bei der Kathode wird das zweite Element mit Wasser und Elektronen kombiniert, ausser bei Lithium-Mangandioxid-Batterie ($Li-MnO_2$, da sich das Lithium in der Kathode einlagert).
\end{itemize}

\subsubsection{$Cu-Zn$}

Anode: $Zn \rightarrow Zn^{2\oplus} + 2e^\ominus$\\

Kathode: $Cu^{2\oplus} + 2e^\ominus \rightarrow Cu$

\subsubsection{$Zn-C$ (Zink-Kohle)}

Vorgänger von Alkali-Mangan\\

Anode: $Zn \rightarrow Zn^{2\oplus} + 2e^\ominus$\\

Kathode: $MnO_2 + H_2O + e^\ominus \rightarrow MnO(OH) + OH^\ominus$

\subsubsection{Alkali-$Mn$}

Kali-Lauge als Elektrolyt. Darum $OH$, Nachfolger von Zink-Kohle-Batterie\\

Anode: $Zn + 4OH^\ominus \rightarrow [Zn(OH)_4]^{2\ominus} + 2e^\ominus$\\

Kathode: $MnO_2 + H_2O + 2e^\ominus \rightarrow MnO(OH) +OH^\ominus$

\subsubsection{$Zn$-Luft}

Anode: $Zn \rightarrow Zn^\oplus + 2e^\ominus$\\

Kathode: $O_2 + H_2O + 4e^\ominus \rightarrow 4OH^\ominus$


\subsubsection{$Li-MnO_2$}

Speziell: Einlagerung von Lithium bei Kathode\\

Anode: $Li \rightarrow Li^\oplus + e^\ominus$\\

Kathode: $Li^\oplus + MnO_2 + e^\ominus \rightarrow LiMnO_2$\\

Nicht aufladbar, weil: $Li^\oplus + Mn(III)O_2 \rightarrow Mn(IV)O_2 + Li^\oplus + e^\ominus$ nicht möglich.

\subsection{Akkus}

Beim Laden Reaktionen umkehren.

\subsubsection{Blei-Akku}

Häufig Sulfat-Ion $SO_4^\ominus$ in Lösung.\\

Anode: $Pb \rightarrow Pb^{2\oplus}+2e^\ominus$\\

Kathode: $PbO_2 + 4 H^\oplus + 2e^\ominus \rightarrow Pb^{2\oplus} + 6H_2O$

\subsubsection{Lithium-Ionen-Akku}

Graphit spielt nur bei der Anode eine Rolle.\\

Anode: $Li_2-Graphit \rightarrow  Graphit + 2Li^\oplus + 2e^\ominus$\\

Kathode: $MnO_2+2Li^\oplus + 2e^\ominus \rightarrow Li2MnO_2$

\subsection{Brennstoffzelle}

% ein galv element

Speziell an Brennstoffzellen: Es wird Wasserstoff $H_2$ und Sauerstoff $O_2$ hinzugeführt.

\begin{itemize}
	\item Bei jeder Anode spielt ein Brennstoff und ein Elektrolyt eine Rolle,
	\item Oxidationsmittel ist bei Brennstoffzellen immer Sauerstoff $O_2$
	\item Reduktion bei Kathode ist immer eine Verbrennung vom Produkt der Anode.
\end{itemize}

\begin{definition}[Brennstoff]
	Wird verbrannt (?).
\end{definition}

\begin{definition}[Elektrolyt]
	Trennt Elektroden.
\end{definition}

\subsubsection{PEM (Proton Exchange Membrane)}

Brennstoff: $H_2$\\

Elektrolyt: $H_2O$\\

Anode: $H_2 + 2H_2O \rightarrow 2 H_3O^\oplus + 2e^\ominus$\\

Kathode: $4 H_3O^\oplus + 4e^\ominus + O_2 \rightarrow 6 H_2O$

\subsubsection{Alkalische Brennstoffzelle}

$KOH \rightarrow K + OH^\ominus$; $K$: Alkalimetall\\

Brennstoff: $H_2$\\

Elektrolyt: $OH^\ominus$\\

Anode: $H_2 + 2 OH^\ominus \rightarrow 2H_2O+2e^\ominus$\\

Kathode: $2 H_2O + 4e^\ominus + O_2 \rightarrow 4 OH^\ominus$

\subsubsection{Direkt-Methanol Brennstoffzelle}

Spezialfall.\\

Methanol: $CH_3OH$\\

Brennstoff: $CH_3OH$\\

Elektrolyt: $H_3O^\oplus$\\

Anode: $CH_3OH + 7 H_2O \rightarrow CO2 + 6 H_3O^\oplus + 12 e^\ominus$\\

Kathode: $4 H_3O^\oplus + 4e^\ominus + O_2 \rightarrow 6 H_2O$


\subsection{Elektrolyse}

Erzwungene Redoxreaktion. Die Anode ist neuerdings der Pluspol, die Kathode der Minuspol. Anlegen einer Spannung bei der Kathode. Umwandlung elektrischer Energie in chemische Energie.

{\large
	\begin{equation}
		\label{elektrolyse}
		\begin{split}
			Gesamt:\ &2\ H_2O \rightarrow 2\ H_2 + O_2 \\
			Reduktion:\ &2\ H^{+I} \rightarrow 2\ e^\ominus + 2\ H^0 \\
			Oxidation:\ &O^{-II} + 2\ e^\ominus \rightarrow 2\ O^0
		\end{split}
	\end{equation}
}

\ref{elektrolyse}: Beispiel: Elektrolyse von Wasser. \\

\begin{definition}[Galvanisieren]
	TODO ergänzen.
\end{definition}
