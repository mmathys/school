\section{Kohlenwasserstoffe}

\subsection{Alkane, Alkene, Alkine, Aromaten}

\subsubsection{Allgemein}

\begin{definition}[Alkane]
	Weisen ausschliesslich Einfachbindungen auf.
\end{definition}

\begin{definition}[Alkene]
	Weisen mindestens eine Doppelbindung auf.
\end{definition}

\begin{definition}[Alkine]
	Weisen mindestens eine Dreifachbindung auf.
\end{definition}

\begin{definition}[Aromate]
	Ringförmige Moleküle, welche über eine ungerade Anzahl alternierender Doppelbindungen verfügen
\end{definition}

\subsubsection{Wichtige Vertreter}

\begin{definition}[Vertreter Alkane]
	Wasser, Ethanol
\end{definition}

\begin{definition}[Vertreter Alkene]
	Werden verwendet für Synthetisierung von Alkohol, Glycerin.
\end{definition}

\begin{definition}[Vertreter Alkine]
	Fungizide
\end{definition}

\begin{definition}[Vertreter Aromate]
	Aminosäuren wie z. B. Phenylalnin. \\
	
	\setatomsep{2.2em}
	\chemfig[][scale=0.8]{HO-[:30](=[:90]O)-[:-30](-[:-90]NH2)-[:30]-[:-30]*6(-=-=-=)}
\end{definition}

\subsubsection{Aromate}

\begin{definition}[Aromat]
	(Zyklische) Kohlenwasserstoffe, die ein delokalisiertes Elektronensystem besitzen.
\end{definition}

\begin{definition}[Hückel-Regel]
	Aromatische Moleküle mit 4n + 2 Elektronen enthalten ein delokalisiertes Elektronensystem.
\end{definition}

\begin{definition}[Benzol]
	$C_6H_6$, Ring von 6 $C$-Atomen mit einem delokalisiertem Elektronensystem. \\
	
	\chemfig{**6(------)}
\end{definition}

\begin{definition}[Delokalisiertes Elektronensystem]
	Elektronen verteilen sich gleichmässig über alle Kohlenstoffatome hinweg.
\end{definition}

\begin{definition}[Kekulé-Formel]
	In Wirklichkeit nicht so. Delokalisiertes Elektronensystem ist aktuell; $\Delta H^0_R$ wäre viel zu krass (viel zu klein, ist Wirklichkeit etwas grösser, d. H. weniger Energie freigesetzt).
	\leavevmode \\
	
	\chemfig{*6(-=-=-=)}
\end{definition}

\subsection{Isomere}

\begin{definition}[Chiralität]
	Ein $C$-Atom hat vier verschiedene Bindungspartner. Chirale Atome in einem Molekül sind Voraussetzung für Enantiomere (1 chirales Atom) und Diastereomere (mindestens 2 chirale Atome).
\end{definition}

\begin{itemize}
	\item \textbf{Konstitutionsisomere}: Moleküle haben gleiche Summenformel, aber andere Verknüpfung von Atomen, andere Namen.
	\item \textbf{Stereoisomere}: Moleküle haben gleiche Summenformel, gleiche Verknüpfung und gleichen Namen, unterschiedliche räumliche Anordnung.
		\subitem \textbf{Konfigurationsisomere}: 
		\subitem $\rightarrow$ \textbf{Enantiomere}: Verhalten sich wie Spiegelbilder, weisen aber keine Symmetrieebene auf. 1 chirales Atom.
		\subitem $\rightarrow$ \textbf{Diastereomere}: Mindestens 2 chirale Atome.
		\subitem $\rightarrow$ \textbf{cis-trans-Isomere}: an einer $C=C$-Doppelbindung sind je zwei verschiedene Substituenten entgegengesetzt (trans) oder auf einer Seite (cis) gebunden.
		\subitem \textbf{Konformationsisomere}: Können durch Rotation ineinander überführt werden.
\end{itemize}

\subsection{Nomenklatur}

Beispiel: \textit{2-Methylpropanol}

\subsubsection{Vorgehen}

\begin{itemize}
	\item \textit{Stammname}: Längste C-Kette ergibt den Stammnamen. So nummerieren, dass die längste Kette gewählt wird und die Substituenten eine möglichst kleine Zahl besitzen
	\item \textit{Funktionelle Gruppen}: Funktionelle Gruppen bestimmen
	\item \textit{Substituenten}: Substituenten bestimmen
	\item \textit{Doppel- und Dreifachbindungen}: Doppel- und Dreifachbindungen mit -en und -in bezeichnen.
\end{itemize}

Format: Substituenten, Stammname, Zwei- / Dreifachbindungen, Funktionelle Gruppe.
\subsubsection{Stammnamen}

\begin{center}
	\begin{tabular}{ l  l  l }
		\textbf{Zahl} & \textbf{Stammname} & \textbf{Zahlwort} \\
		1 & Meth & \\
		2 & Eth & Di\\
		3 & Prop & Tri\\
		4 & But & Tetra\\
		5 & Pent & Penta\\
		6 & Hex & Hexa\\
		7 & Hept & Hepta\\
		8 & Oct & Octa\\
		9 & Non &Nona\\
		10 & Dec &Deca\\
	\end{tabular}
\end{center}

\subsubsection{Substituenten}

Bei Einfachbindung: \{Name\}-yl anhängen.

Bei Doppelbindung:  \{Name\}-ylen anhängen.

Bei Dreifachbindung:  \{Name\}-ylidin anhängen.

Bei Bei Cycloverbindungen:  Cyclo-\{Name\} einfügen.

\begin{center}
	\begin{tabular}{ l  l  l }
		\textbf{Substituent} & \textbf{Name} \\
		$-CH_3$ & Methyl \\
		$-C_2H_5$ & Ethyl \\
		$-C_nH_{2n}$ & ?yl \\
		$-F$ & Fluor \\
		$-Cl$ & Chlor \\
		$-Br$ & Brom \\
		$-I$ & Iod \\
		$-NH_2$ & Amino \\
		$-OH$ & Hydroxy \\
		$-NO_2$ & Nitro \\
		$-SH$ & Mercapto \\
	\end{tabular}
\end{center}

\begin{note}
	Substituenten, Funktionelle Gruppen und Doppel-/Dreifachbindungen werden immer mit dem Format \\
	
	$\{a_0,...,a_n\}-\{Zahlwort\}\{Name\}$, z. B. 3,4-dien \\
	
	bezeichnet. 
	$a$: Der Ort auf der C-Kette, wo gebunden.
\end{note}

\subsubsection{Funktionelle Gruppen}

\begin{definition}[Halogene Wasserstoffe]
	Enthalten mindestens ein Halogenatom $F,\ Cl,\ Br,\ I$. Struktur: $R-X$
	
	Als Name für die funktionelle Gruppe dient der Name des Halogenatoms. 
\end{definition}

\begin{definition}[Reihenfolge der Verbindungen]
	\textit{CCCAKATE} \\
	
	\setatomsep{2.2em}
	\begin{tabularx}{.5\textwidth}{X l l}
		\textbf{Name + Benennung} & \textbf{Formel} \vspace{2em}\\
		
		\vspace{-2em} 
		Carbonsäure
		
		 \textit{\{Stamm\}säure}& 
		\chemfig[][scale=.8]{C(=[:90]\lewis{13,O})(-[:-150]R)-[:-30]OH} \vspace{2em} \\
		
		\vspace{-2em} 
		Carbonsäureester
		
		\textit{\{Stamm\}säure-\{R'\}ester} & 
		\chemfig[][scale=.8]{C(=[:90]\lewis{13,O})(-[:-150]R)-[:-30]\lewis{57,O}-[:30]R'} \vspace{2em} \\
		
		\vspace{-2em} 
		Carbonsäureamide
		
		\textit{\{Stamm\}säure-N-\{R'\}yl-N-\{R''\}ylamid}
		
		\textit{\{Stamm\}säure-N-N-di\{R'\}ylamid} & 
		\chemfig[][scale=.8]{C(=[:90]\lewis{13,O})(-[:-150]R)-[:-30]NH_2} \vspace{2em} \\
	
		\vspace{-2em} 
		Aldehyde
		
		\textit{\{Stamm\}al}& 
		\chemfig[][scale=.8]{C(=[:90]\lewis{13,O})(-[:-150]R)-[:-30]H} \vspace{2em} \\
		
		\vspace{-2em} 
		Ketone
		
		\textit{\{Stamm\}on} &
		\chemfig[][scale=.8]{C(=[:90]\lewis{13,O})(-[:-150]R)-[:-30]R'} \vspace{2em} \\
		
		\vspace{-2em} 
		Alkohole
		
		\textit{\{Stamm\}ol} &
		\chemfig[][scale=.8]{R-OH} \vspace{2em} \\
		
		\vspace{-2em} 
		Thiole
		
		\textit{\{Stamm\}thiol} &
		\chemfig[][scale=.8]{R-SH} \vspace{2em} \\
		
		\vspace{-2em} 
		Amine
		
		\textit{\{Stamm\}ylamin} &
		\chemfig[][scale=.8]{R-NH_2} \vspace{2em} \\
		
		\vspace{-2em} 
		Ether
		
		\textit{\{R'\}yl-\{R''\}ylether} &
		\chemfig[][scale=.8]{R-[:30]\lewis{13,O}-[:-30]R} \vspace{2em} \\
	\end{tabularx}
\end{definition}

\subsubsection{Cycloalkane}

Stamm: \textit{Cyclo\{Stamm\}\{en/in\} }

Substituent: \textit{Cyclo\{Gruppe\}yl-\{Stamm\}\{en/in\} } \\

\setatomsep{2.2em}
\chemfig{*6(-=-=-=)} (Benzol) \ \
\ \chemfig{*6(------)} (Cyclohexan)

\subsubsection{Aromate}

Als Stamm: \{Substituenten\}benzol

Als Substituent: Phenyl-\{Stamm\} \\

\setatomsep{2.2em}
\chemfig{**6(----(-[:90]C(=[:150]\lewis{24,O})-[:30]OH)--)} %benzoesäure 
(Phenylmethansäure)
\chemfig{**6(----(-[:90]OH)--)} %phenol
(\textbf{Phenol}) 
\chemfig{**6(----(-[:90]NH_2)--)} %phenol
(Phenylamin oder Anilin) 







