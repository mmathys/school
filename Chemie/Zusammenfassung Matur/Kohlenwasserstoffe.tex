\section{Kohlenwasserstoffe}

\subsection{Alkane, Alkene, Alkine, Aromaten}

\subsubsection{Allgemein}

\begin{definition}[Alkane]
	Weisen ausschliesslich Einfachbindungen auf.
\end{definition}

\begin{definition}[Alkene]
	Weisen mindestens eine Doppelbindung auf.
\end{definition}

\begin{definition}[Alkine]
	Weisen mindestens eine Dreifachbindung auf.
\end{definition}

\begin{definition}[Aromate]
	Ringförmige Moleküle, welche über eine ungerade Anzahl alternierender Doppelbindungen verfügen
\end{definition}

\subsubsection{Wichtige Vertreter}

\begin{definition}[Vertreter Alkane]
	Wasser, Ethanol
\end{definition}

\begin{definition}[Vertreter Alkene]
	Werden verwendet für Synthetisierung von Alkohol, Glycerin.
\end{definition}

\begin{definition}[Vertreter Alkine]
	Fungizide
\end{definition}

\begin{definition}[Vertreter Aromate]
	Aminosäuren wie z. B. Phenylalnin. \\
	
	\setatomsep{2.2em}
	\chemfig[][scale=0.8]{HO-[:30](=[:90]O)-[:-30](-[:-90]NH2)-[:30]-[:-30]*6(-=-=-=)}
\end{definition}

\subsection{Isomere}

\begin{definition}[Chiralität]
	TODO
\end{definition}

TODO isomerie

\subsection{Nomenklatur}

Beispiel: \textit{2-Methylpropanol}

\subsubsection{Vorgehen}

\begin{itemize}
	\item \textit{Stammname}: Längste C-Kette ergibt den Stammnamen. So nummerieren, dass die längste Kette gewählt wird und die Substituenten eine möglichst kleine Zahl besitzen
	\item \textit{Funktionelle Gruppen}: Funktionelle Gruppen bestimmen
	\item \textit{Substituenten}: Substituenten bestimmen
	\item \textit{Doppel- und Dreifachbindungen}: Doppel- und Dreifachbindungen mit -en und -in bezeichnen.
\end{itemize}

Format: Substituenten, Stammname, Zwei- / Dreifachbindungen, Funktionelle Gruppe.
\subsubsection{Stammnamen}

\begin{center}
	\begin{tabular}{ l  l  l }
		\textbf{Zahl} & \textbf{Stammname} & \textbf{Zahlwort} \\
		1 & Meth & \\
		2 & Eth & Di\\
		3 & Prop & Tri\\
		4 & But & Tetra\\
		5 & Pent & Penta\\
		6 & Hex & Hexa\\
		7 & Hept & Hepta\\
		8 & Oct & Octa\\
		9 & Non &Nona\\
		10 & Dec &Deca\\
	\end{tabular}
\end{center}

\subsubsection{Substituenten}

Bei Einfachbindung: \{Name\}-yl anhängen.

Bei Doppelbindung:  \{Name\}-ylen anhängen.

Bei Dreifachbindung:  \{Name\}-ylidin anhängen.

Bei Bei Cycloverbindungen:  Cyclo-\{Name\} einfügen.

\begin{center}
	\begin{tabular}{ l  l  l }
		\textbf{Substituent} & \textbf{Name} \\
		$-CH_3$ & Methyl \\
		$-C_2H_5$ & Ethyl \\
		$-C_nH_{2n}$ & ?yl \\
		$-F$ & Fluor \\
		$-Cl$ & Chlor \\
		$-Br$ & Brom \\
		$-I$ & Iod \\
		$-NH_2$ & Amino \\
		$-OH$ & Hydroxy \\
		$-NO_2$ & Nitro \\
		$-SH$ & Mercapto \\
	\end{tabular}
\end{center}

\begin{note}
	Substituenten, Funktionelle Gruppen und Doppel-/Dreifachbindungen werden immer mit dem Format \\
	
	$\{a_0,...,a_n\}-\{Zahlwort\}\{Name\}$, z. B. 3,4-dien \\
	
	bezeichnet. 
	$a$: Der Ort auf der C-Kette, wo gebunden.
\end{note}

\subsubsection{Funktionelle Gruppen}

\begin{definition}[Halogene Wasserstoffe]
	Enthalten mindestens ein Halogenatom $F,\ Cl,\ Br,\ I$. Struktur: $R-X$
	
	Als Name für die funktionelle Gruppe dient der Name des Halogenatoms. 
\end{definition}

\begin{definition}[Reihenfolge der Verbindungen]
	\textit{CCCAKATE}
	
	\setatomsep{2.2em}
	\begin{tabular}{ l  l }
		\textbf{Name} & \textbf{Formel} \vspace{1em} \\
		
		Carbonsäure (\textit{\{Stamm\}-säure}) & 
		\chemfig[][scale=.8]{C(=[:90]\lewis{13,O})(-[:-150]R)-[:-30]OH}  \vspace{1.8em} \\
		
		
		Carbonsäureester (\textit{\{Stamm\}-säure ester ester ester tester})& 
		\chemfig[][scale=.8]{C(=[:90]\lewis{13,O})(-[:-150]R)-[:-30]\lewis{57,O}-[:30]R'} \vspace{1.8em} \\
		
		Carbonsäureamide & 
		\chemfig[][scale=.8]{C(=[:90]\lewis{13,O})(-[:-150]R)-[:-30]NH_2} \vspace{1.8em} \\
	
		Aldehyde & 
		\chemfig[][scale=.8]{C(=[:90]\lewis{13,O})(-[:-150]R)-[:-30]H} \vspace{1.5em} \\
		
		Ketone &
		\chemfig[][scale=.8]{C(=[:90]\lewis{13,O})(-[:-150]R)-[:-30]R'} \vspace{1.5em} \\
		
		Alkohole &
		\chemfig[][scale=.8]{R-OH} \vspace{1.5em} \\
		
		Thiole &
		\chemfig[][scale=.8]{R-SH} \vspace{1.5em} \\
		
		Amine &
		\chemfig[][scale=.8]{R-NH_2} \vspace{1.5em} \\
		
		Ether &
		\chemfig[][scale=.8]{R-[:30]\lewis{13,O}-[:-30]R} \vspace{1.5em} \\
	\end{tabular}
\end{definition}

TODO aromate, cyclo
