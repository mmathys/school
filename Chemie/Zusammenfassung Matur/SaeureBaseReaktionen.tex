\section{Säure-Base Reaktionen}

\subsection{Definition nach Brönsted}

\textit{Säuren sind Protonenspender, Basen sind Protonenakzeptoren.}

\subsection{Säure-Base Reaktionen (Protolyse)}

\begin{itemize}
	\item Nichtmetalloxide erzeugen saure Lösungen, Metalloxide erzeugen basische Lösungen.
	\item Saure und basische Lösungen leiten Strom: Die Lösungen enthalten Ionen.
	\item Säuren enthalten Wasserstoffatome, Basen enthalten stark elektronegative Atome mit nichtbindenden Elektronenpaaren.
\end{itemize}

\begin{definition}[Protolyse]
	Säure-Base-Reaktion: Ein Stoff gibt ein Proton ab, ein Stoff nimmt ein Proton auf.
\end{definition}

\begin{definition}[Beispiel Säure]
	{\large
	$HCl$
}
\end{definition}

\begin{definition}[Beispiel Säurereaktion]
	{\large
		$HCl + H_2O \rightarrow Cl^\ominus + H_3O^\oplus$
	}
\end{definition}

\begin{definition}[Beispiel Base]
	{\large
		$NH_3$
	}
\end{definition}

\begin{definition}[Beispiel Basereaktion]
	{\large
		 $NH_3 + H_2O \rightarrow NH_4^\oplus + OH^\ominus$
	}
\end{definition}

\begin{definition}[Hydronium-Ion]
	{\large
		$H_3O^\oplus$
	}
\end{definition}

\begin{definition}[Hydroxid-Ion]
	{\large
		$OH^\ominus$
	}
\end{definition}

\begin{definition}[Ampholyte]
	Stoffe, die die Funktion einer Säure und einer Base einnehmen können. Zum Beispiel {\large $H_2O$}
\end{definition}

\begin{definition}[Autoprotolyse von Wasser]\leavevmode \\

{\large
	$H_2O + H_2O \rightarrow H_3O^\oplus + OH^\ominus$	
}

\end{definition}

\subsection{pH Berechnungen}

\subsubsection{Allgemein}

Das Massenwirkungsgesetz (\ref{mwg}) auf Autoprotolyse anwenden:

{\large
	\begin{equation}
	K_c=\frac{c^c(C) \cdot c^d(D)}{c^a(A) \cdot c^b(B)}=\frac{c(H_3O^\oplus) \cdot c(OH^\ominus)}{c^2(H_2O)}
	\end{equation}
}

Man kann nun annehmen, dass $c(H_2O)$ konstant ist im Reaktionssystem, wenn nur kleine Mengen an Säuren und Basen reagieren $\rightarrow$ \textit{Ionenprodukt}.

\begin{definition}[Ionenprodukt]\leavevmode \\
	
	{\large
		\begin{equation}
		\begin{split}
		K_w & = K_c \cdot c^2(H_2O) \\
		       & = c(H_3O^\oplus) \cdot c(OH^\ominus) \\
		       & = K_S \cdot K_W \\
		       & = 10^{-14} \ M^2
		\end{split}
		\end{equation}
		
		$K_w=K_c \cdot c^2(H_2O)=c(H_3O^\oplus) \cdot c(OH^\ominus)=10^{-14}\ M^2$
	}
	(Bei $25\degree C$)
	
\end{definition}

\begin{definition}[Saure Lösung]
	{\large
		$c(H3O^\oplus) > 10^{-7} M$
	}
\end{definition}

\begin{definition}[Neutrale Lösung]
	{\large
		$c(H3O^\oplus) = 10^{-7} M$
	}
\end{definition}

\begin{definition}[Basische Lösung]
	{\large
		$c(H3O^\oplus) < 10^{-7} M$
	}
\end{definition}

\begin{definition}[pH-Wert]
	Negativer 10-er Logarithmus der Konzentration der Hydroniumionen.
	
	{\large
		\begin{equation}
			\begin{split}
				pH		&=-log_{10}\ c(H_3O^\oplus) \\
						  &= 14-pOH
			\end{split}
		\end{equation}
	}
\end{definition}

\begin{definition}[pOH-Wert]
	Negativer 10-er Logarithmus der Konzentration der Hydroxidionen.
	
	{\large
		\begin{equation}
			\begin{split}
				pOH&=-log_{10}\ c(OH^\ominus) \\
				&=14-pH
			\end{split}
		\end{equation}
	}
\end{definition}

Skala reicht von 0 bis 14. 0 ist sauer, 7 ist neutral, 14 ist basisch.

\begin{definition}[Säurekonstante]
	Bei einer vorhandenen Protolyse, bei der eine Säure mit Wasser reagiert, wird die Gleichgewichtskonstante ($K_c$) mit der Konzentration des Wassers ($c(H_2O)$) multipliziert, da sich diese ja konstant verhält.
	
	{\large
		\begin{equation}
		\begin{split}
		K_S & =K_C \cdot c(H_2O) = \frac{c(H_3O^\oplus) \cdot c(X^\ominus)}{c(HX)} \\
		pK_S & = -log_{10}\ K_S
		\end{split}
		\end{equation}
	}
	
	Hoher $K_S$-Wert oder niedriger $pK_S$-Wert: Vollständige Protonenabgabe.
	
	Tiefer $K_S$-Wert oder hoher $pK_S$-Wert: Unvollständige Protonenabgabe.
\end{definition}

\begin{definition}[Basenkonstante]
	Das gleiche wie Säurekonstante, einfach für basische Reaktionen.
	
	{\large
		\begin{equation}
		\begin{split}
		K_B & =K_C \cdot c(H_2O) = \frac{c(OH^\ominus) \cdot c(HB^\oplus)}{c(B)} \\
		pK_B & = -log_{10}\ K_B
		\end{split}
		\end{equation}
	}
	
	Hoher $K_B$-Wert oder niedriger $pK_B$-Wert: Vollständige Protonenaufnahme.
	
	Tiefer $K_B$-Wert oder hoher $pK_B$-Wert: Unvollständige Protonenaufnahme.
\end{definition}

{\large
\begin{equation}
	pK_S + pK_B = 14.
\end{equation}
}

\begin{definition}[Sehr starke Säure]
	Sie geben ihre Protonen vollständig ab. \\
	{\large $pK_S < 0$}
\end{definition}

\begin{definition}[Sehr starke Base]
	Sie nehmen  Protonen vollständig auf. \\
	{\large $pK_B < 0$}
\end{definition}

\subsubsection{pH-Berechnung mit starken Säuren/Basen}

Annahme: Bei starken Säuren oder starken Basen reagiert die Säure oder die Base völlig. \\

Für $HCl\ (pK_S=-6; \ c(HCl) = 10^{-2} M)$:\\

$HCl + H_2O \rightarrow H_3O^\oplus + Cl^\ominus$\\

$c_{GGW}(H_3O^\oplus) = c_0(HCl); \ K_S = 10^{-2}; \ pK_S = 2$\\

{\large
	\begin{equation}
	\label{säure}
		\begin{split}
			 K_S=c_{GGW}(H_3O^\oplus) &= c_0(HX)
		\end{split}
	\end{equation}	
}
\ref{säure}: Starke Säure\\

{\large
	\begin{equation}
	\label{base}
	\begin{split}
	K_B=c_{GGW}(OH^\ominus) &= c_0(B)
	\end{split}
	\end{equation}	
}
\ref{base}: Starke Base
\subsubsection{pH-Berechnung mit schwachen Säuren/Basen}

\begin{alignat*}{5}
	&HX \ \ \ +\ \ \ &&H_2O\ \ \rightarrow \ \ &&H_3O^\oplus \ \ \ +\ \ \ &&X^\ominus && \\
	&c_0 - x && konst. && x && x &&
\end{alignat*}
\begin{alignat*}{5}
	&B \ \ \ +\ \ \ &&H_2O\ \ \rightarrow \ \ &&OH^\ominus \ \ \ +\ \ \ &&HB && \\
	&c_0 - x && konst. && x && x &&
\end{alignat*}
{\large
	\begin{equation}
		K_{\{S|B\}} = \frac{x^2}{c_0-x}
	\end{equation}	
}
{\large
	\begin{equation}
	x^2 + x\cdot K_{\{S|B\}} - c_0 \cdot K_{\{S|B\}} = 0
	\end{equation}
}
{\large
	\begin{equation}
	\begin{split}
		\{pH\ |\ pOH\}&=-log_{10}\ x \\
		&=14-\{pOH\  |\  pH\}
	\end{split}
	\end{equation}
}
\subsection{Puffer}
Eine Lösung, die den pH-Wert in einem bestimmten Bereich konstant hält.
Ein Paar aus einer schwachen Säure $HPu$ und einer schwachen Base $Pu^\ominus$. Bsp: $NH_4$ / $NH_3^\ominus$
{\large
\begin{equation}
	\label{puffergleichung}
	HPu + H_2O \rightarrow Pu^\ominus + H_3O^\oplus	
\end{equation}
}
\ref{puffergleichung}: Puffergleichnung
{\large
	\begin{equation}
	\label{pufferausgleich}
	\begin{split}
	Pu^\ominus + H_3O^\oplus \rightarrow HPu + H_2O \\
	HPu + OH^\ominus \rightarrow Pu^\ominus + H_2O
	\end{split}
	\end{equation}
}

\ref{pufferausgleich}: Ausgleich der Puffergleichung von Hinzugabe von Hydronium- und Hydroxidionen.\\

{\large
	\begin{equation}
	\begin{split}
		\label{pkspuffer}
		K_S = \frac{c(Pu^\ominus) \cdot c(H_3O^\oplus)}{HPu}\\
		pK_{S,\ Puffer} = pH - log_{10}\ \frac{c(Pu^\ominus)}{c(HPu)}
	\end{split}
	\end{equation}
}

\ref{pkspuffer}: $K_S$-Wert

{\large
	\begin{equation}
	\begin{split}
	\label{puffergl}
	pH= pK_{S,\ Puffer} + log_{10}\ \frac{c(Pu^\ominus)}{c(HPu)}
	\end{split}
	\end{equation}
}

\ref{puffergl}: Puffergleichung

\subsubsection{Existenzgebiete von Teilchen}

$pH = pK_s \Rightarrow \dfrac{c(Pu^\ominus)}{c(HPu)} = 1$: Je 50\% Säure und Säurerest in der Pufferlösung.\\

$pH > pK_s \Rightarrow \dfrac{c(Pu^\ominus)}{c(HPu)} > 1$: Mehr Säure, weniger Säurerest in der Pufferlösung.\\

$pH < pK_s \Rightarrow \dfrac{c(Pu^\ominus)}{c(HPu)} < 1$: Weniger Säure, mehr Säurerest in der Pufferlösung.

\subsection{Neutralisationen}

Ziel der Neutralisation: Durch Zugabe von Säure oder Base den Stoff zu einem neutralen pH-Wert (pH=7) zu führen.

Zweck: Bestimmen der Konzentrationen von Säuren und Basen. Siehe Titration.

\subsection{Titrationen}

\subsubsection{Säure-Base Titration}

Eine saure Lösung wir mit einer basischen Lösung neutralisiert (oder umgekehrt). Ziel: Konzentration eines Stoffes herausfinden durch Hinzugabe von Masslösung und Benutzen eines pH-Indikators.

\begin{definition}[Äquivalenzpunkt]
	Bei einem ÄP: pH-Wert ist bei 7, in der Mitte des pH-Sprungs, die Lösung ist neutral. In diesem Fall ist
	
	{\large
		\begin{equation}
			n(S\ddot{a}ure) = n(Massl\ddot{o}sung)
		\end{equation}	
	}
	
	y-Achse: pH; x-Achse: n(Masslösung) oder ml M oder ml mol/l.
\end{definition}

Da $n=c \cdot V$. $c$ ist die Konzentration des Masslösung, $V$ die verbrauchte Menge der Masslösung. $n_S$ und $n_M$ gleichsetzen, \textbf{Achtung}: Für $V_s$ den Anfangswert des Volumens nehmen, nicht der Wert mit der Masslösung drinnen!

{\large
	\begin{equation}
	\label{titration}
	V_{S\ddot{a}ure}\ \cdot \ c(S\ddot{a}ure) \ \cdot \ z_{S\ddot{a}ure} = V_{Base} \ \cdot \ c(Base) \ \cdot \ z_{Base}
	\end{equation}	
}

\ref{titration}: Grundgleichung der Titration.
