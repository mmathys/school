\section{Kunststoffe}

\subsection{Polymere}

\begin{itemize}
	\item \textit{Polymere} bestehen aus \textit{Makromolekülen}.
	\item \textit{Makromoleküle} bestehen aus \textit{Monomeren}.
\end{itemize}

Grund, dass Polymere schmelzen und nicht ruckartig verflüssigen: Van der Waals-Kräfte und grosse Oberfläche der Makromolekülen.

\subsection{Einteilung}

\subsubsection{Einteilung nach Struktur}

\begin{center}
	\begin{tabular}{l l l}
		\textbf{Name} & \textbf{Verzweigung} & \textbf{Andere Eigenschaften} \\
		Thermoplast & linear & Schmelzbar \& weich \\
		Elastomer & 2-D & Nicht schmelzbar \\
		Duroplast & 3-D & Nicht schmelzbar \& hart
	\end{tabular}
\end{center}

\subsubsection{Einteilung nach Syntheseart}

\begin{center}
	\begin{tabular}{l l l}
		\textbf{Synthese} & \textbf{Produkte} & \textbf{Nebenprodukte} \\
		Polymerisation & Polymer & Nein \\
		Polykondensation & Polyamide, Polyester & Ja, $H_2O$\\
		Polyaddition & Polyaddukt & Nein
	\end{tabular}
\end{center}

\subsection{Polymerisation}

\begin{definition}[Radikalische Polymerisation]
	Kettenreaktion, bei der sich \textit{Monomere mit einer Doppelbindung} zu einem Polymer verbinden. \textit{Radikale} dienen hier als "Katalysator", die die Doppelbindung aufheben. \\
	
	%
	% Radikale Polymerisation
	%
	
	\setatomsep{2.2em}
	\textbf{Voraussetzungen}
	
	\chemfig{C(-[:150]H)(-[:210]H)=C(-[:30]H)(-[:-30]H)} Monomer (Ethen)\\
	
	\chemfig{\lewis{0.,R}}\hspace{.5em} Startradikal (Aus Azobiisobutyronitril) \\%Startradikale Azobiisobutyronitril
	
	\textbf{Wichtige Monomere}\\
	
	\chemfig{=}\\
	
	Ethen bei Plastiktaschen\\
	
	\chemfig{=-[:-30]Cl} \\
	
	Polyvinylchlorid\\
	
	\chemfig{F-[:-30](-[:-150]F)(=(-[:30]F)(-[:-30]F))} \\
	
	Polytetrafluorethen bei Pfannen\\
	
	\chemfig{=[:-30]F} \\
	
	Polypropen\\
	
	\chemfig{=-[:-30]*6(-=-=-=)} \\
	
	Polystyrol bei Styropor\\
	
	\chemfig{=-[:-30]~[:-30]N} \\
	
	Polyacrylnitril\\
	
	\chemfig{-[:-150](=[:180])-[:-30](=O)-[:-150]O-[:-30]}\\
	
	Polymethylmethacrylat bei Plexiglas
	
\end{definition}

\subsection{Polykondensation}

% ist synthese

\subsubsection{Carbonsäureamidbildung}

%carbonsäureamidbildung = entstehung polyamide

\subsubsection{Esterbildung}

%Besterbildung = entstehung polyester

\subsection{Polyaddition}

% ist synthese

\subsection{Spezielle Materialien}


\subsubsection{PLA (Polylactid)}
%% PLA (seiet 8)

\subsubsection{PVA (Polyvinylalkohol)}
%% PVA (löslich seite 13)

\subsubsection{Superabsorber}
